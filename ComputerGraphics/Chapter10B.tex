\section{卷积核}
% 在这一节,我们会列出许多常用的卷积核,并讨论它们的性质。

\subsection{卷积核列表}
\begin{BoxFormula}[Box卷积核]
    Box卷积核是指
    \begin{Equation}
        f_{box}(x)=\begin{cases}
            1/2,&0\leq |x|<1\\
            0,&\te{otherwise}
        \end{cases}
    \end{Equation}
\end{BoxFormula}

\begin{BoxFormula}[Tent卷积核]
    Tent卷积核是指
    \begin{Equation}
        f_{tent}(x)=\begin{cases}
            1-|x|,&0\leq |x|<1\\
            0,&\te{otherwise}
        \end{cases}
    \end{Equation}
\end{BoxFormula}

\begin{BoxFormula}[B-Spline Cubic卷积核]
    B-Spline Cubic卷积核是指
    \begin{Equation}
        f_{B}(x)=\frac{1}{6}\begin{cases}
            -3(1-|x|)^3+3(1-|x|)^2+3(1-|x|)+1,&0\leq |x|<1\\
            (2-|x|)^3,&1\leq |x|<1\\
            0,&\te{otherwise}
        \end{cases}
    \end{Equation}
\end{BoxFormula}

\begin{BoxFormula}[Catmull-Rom Cubic卷积核]
    Catmull-Rom Cubic卷积核是指
    \begin{Equation}
        f_C(x)=\frac{1}{2}\begin{cases}
            -3(1-|x|)^3+4(1-|x|)^2+(1-|x|),&0\leq |x|<1\\
            (2-|x|)^3-(2-|x|)^2,&1\leq |x|<1\\
            0,&\te{otherwise}
        \end{cases}
    \end{Equation}
\end{BoxFormula}

\begin{BoxFormula}[Mitchell-Netravali Cubic卷积核]
    Mitchell-Netravali Cubic卷积核是指
    \begin{Equation}
        f_M(x)=\frac{1}{18}\begin{cases}
            -21(1-|x|)^3+27(1-|x|)^2+9(1-|x|)+1,&0\leq |x|<1\\
            7(2-|x|)^3-6(2-|x|)^2,&1\leq |x|<2\\
            0,&\te{otherwise}
        \end{cases}
    \end{Equation}
\end{BoxFormula}

\begin{BoxFormula}[Gaussian卷积核]
    Gaussian卷积核是指
    \begin{Equation}
        f_{g,\sigma}(x)=\frac{1}{\sigma\sqrt{2\pi}}\exp(-\frac{x^2}{2\sigma^2})
    \end{Equation}
\end{BoxFormula}

在\cref{fig:卷积核的曲线图}中展示了上述卷积核的图像。卷积核具有不同峰值,但都是归一化的。
\begin{itemize}
    \item Tent卷积核、Box卷积核的半径为$1$。
    \item B-Spline卷积核、Mitchell-Netravali卷积核、Catmull-Rom卷积核的半径为$2$。
    \item Gaussian卷积核的半径是无穷大的,但可以在$3\sigma$的位置截断获得有限的半径。
\end{itemize}
\begin{Figure}[卷积核的曲线图]
    \figuresub[较简单的卷积核]{\includegraphics[scale=0.8]{Chapter10B_01a.fig.pdf}}    
    \figuresub[较复杂的卷积核]{\includegraphics[scale=0.8]{Chapter10B_01b.fig.pdf}}    
\end{Figure}

需要说明的是,B-Spline Cubic和Catmull-Rom Cubic的名称就是来自于\cref{chap:参数曲线}中相应曲线的混合函数(Catmull-Rom Cubic是$\alpha=0$的Cardinal Cubic),它们共享了相同的表达式。

除此之外,Mitchell-Netravali Cubic卷积核可以表达为
\begin{Equation}
    f_M(x)=\frac{1}{3}f_B(x)+\frac{2}{3}f_C(x)
\end{Equation}

有时需要缩放一个卷积核,以下是相应的公式。对于Gaussian核,改变$\sigma$等效于缩放。
\begin{BoxFormula}[卷积核的缩放]
    卷积核$f(x)$可以通过下式缩放为$f_s(x)$
    \begin{Equation}
        f_s(x)=\frac{f(x/s)}{s}
    \end{Equation}
\end{BoxFormula}

\subsection{卷积核的性质}
这里的一个问题是,上述介绍的都是连续的卷积核,它们怎么和离散的信号相互作用呢?\goodbreak

第一种方法是,将离散信号视为相应幅度的狄拉克梳,并与卷积核做连续卷积。这在实质上等价于,在每一个整数点位置上复制一个相应幅度的卷积核曲线,并将它们简单叠加起来。
\begin{Equation}
    (a*f)(x)=\Sum[i'] a[i']f(x-i')
\end{Equation}

第二种方法是,将连续卷积核通过取整数点的方式当作离散卷积核
\begin{Equation}
    (a*f)[i]=\Sum[i'] a[i']f(i-i')
\end{Equation}

两种方法分别会给出函数和数列的结果。在这一小节,将应用第一种方法研究核的性质。

\begin{Table}[卷积核的性质]!!
    \begin{tblr}
    {
        colspec={X[c]X[c]},
        cell{odd[2-Z]}{1}={c=2}{halign=c},
        hline{even[3-Y]}={\linesplit},
        row{even[2-Z]}={abovesep=10pt},
        row{odd[2-Z]}={belowsep=4pt},
    }
        一般信号&常值信号\\
        \includegraphics[scale=0.8]{Chapter10B_02a.fig.pdf}&
        \includegraphics[scale=0.8]{Chapter10B_02i.fig.pdf}\\*
        信号$a[i]$&\\
        \includegraphics[scale=0.8]{Chapter10B_02b.fig.pdf}&
        \includegraphics[scale=0.8]{Chapter10B_02j.fig.pdf}\\*
        Box卷积核&\\
        \includegraphics[scale=0.8]{Chapter10B_02c.fig.pdf}&
        \includegraphics[scale=0.8]{Chapter10B_02k.fig.pdf}\\*
        Tent卷积核&\\
        \includegraphics[scale=0.8]{Chapter10B_02d.fig.pdf}&
        \includegraphics[scale=0.8]{Chapter10B_02l.fig.pdf}\\*
        Gaussian卷积核($\sigma=0.5$)&\\
        \includegraphics[scale=0.8]{Chapter10B_02e.fig.pdf}&
        \includegraphics[scale=0.8]{Chapter10B_02m.fig.pdf}\\*
        Gaussian卷积核($\sigma=1.0$)&\\
        \includegraphics[scale=0.8]{Chapter10B_02f.fig.pdf}&
        \includegraphics[scale=0.8]{Chapter10B_02n.fig.pdf}\\*
        Catmull-Rom Cubic卷积核&\\
        \includegraphics[scale=0.8]{Chapter10B_02g.fig.pdf}&
        \includegraphics[scale=0.8]{Chapter10B_02o.fig.pdf}\\*
        Mitchell-Netravali Cubic卷积核&\\
        \includegraphics[scale=0.8]{Chapter10B_02h.fig.pdf}&
        \includegraphics[scale=0.8]{Chapter10B_02p.fig.pdf}\\*
        B-Spline Cubic卷积核&\\
    \end{tblr}
\end{Table}

在\cref{tab:卷积核的性质},我们展示了上述卷积核与视为狄拉克梳的离散信号的卷积结果,这产生了一条贯穿离散信号的连续曲线,实践上这一方法常被用于重采样。例如,这里的离散信号总计有$11$个采样点,假如最终需要$7$个或$15$个采样点的信号,显然简单的删掉或复制一部分采样点得到的结果并不理想。此时,就可以通过这种方式先还原出连续曲线,随后再根据需要重新采样。

在\cref{tab:卷积核的性质}中可以看出不同卷积核产生的连续曲线是有所不同的,这反映了卷积核的性质
\begin{itemize}
    \item 插值(Interpolating):曲线是否通过每个采样点。
    \item 过冲(Overshoot):曲线是否位于采样点的凸包内。
    \item 无纹波(Ripple Free):曲线在常输入信号下是否也为恒定值。
\end{itemize}
Box卷积核产生矩形波,矩形波的电平高度有相邻两个采样点的高度均值决定。Tent卷积核产生穿过所有采样点的折线段。Catmull-Rom Cubic卷积核与B-Spline Cubic卷积核和对应的曲线类似,分别满足“有过冲,插值”和“无过冲,非插值”。Mitchell-Netravali卷积核作为两者的混合,则为“有过冲,非插值”。Gaussian卷积核可以提供最为平滑的效果,其同样也是“无过冲,非插值”。上述所有卷积核在标准情形下(非缩放)都是无纹波的,然而,缩放后就未必了。我们可以看到$\sigma=0.5$的Gaussian核(对于Gaussian核而言改变$\sigma$等效于进行缩放)在常数输入信号下产生的曲线并非能保持定值,而是会在采样点间上下波动。

\subsection{二维卷积核}
我们当然可以定义新的二维卷积核。但如果有已知的一维卷积核,可以这样构造二维卷积核
\begin{BoxFormula}[可分离的二维卷积核]
    通过一维卷积核$b_1[i]$可以构造二维卷积核$b_2[i,j]$
    \begin{Equation}
        b_2[i,j]=b_1[i]b_1[j]
    \end{Equation}
\end{BoxFormula}
在\cref{fig:二维卷积核}中,我们展示了\cref{subsec:卷积核列表}中提到的所有卷积核对应的二维卷积核的图像。
% 请注意,由于$b_1[0]$未必等于$1$,在大部分情况下$b_2[i,0]\neq b_1[i]$及$b_2[0,j]\neq b_1[j]$,即通过上述方式构造出的二维卷积核,沿着$x$轴或$y$轴得到的,并不一定是构造其的一维卷积核!

\begin{Figure}[二维卷积核]
    \figuresub[Box]{\includegraphics[scale=0.85]{Chapter10B_03a.fig.pdf}\hspace*{0.5cm}}{0.48}
    \figuresub[Tent]{\includegraphics[scale=0.85]{Chapter10B_03b.fig.pdf}\hspace*{0.5cm}}{0.48}\vspace{16pt}

    \figuresub[Gaussian]{\includegraphics[scale=0.85]{Chapter10B_03c.fig.pdf}\hspace*{0.5cm}}{0.48}
    \figuresub[B-Spline Cubic]{\includegraphics[scale=0.85]{Chapter10B_03f.fig.pdf}\hspace*{0.5cm}}{0.48}\vspace{16pt}

    \figuresub[Mitchell-Netravali Cubic]{\includegraphics[scale=0.85]{Chapter10B_03e.fig.pdf}\hspace*{0.5cm}}{0.48}
    \figuresub[Catmull-Rom Cubic]{\includegraphics[scale=0.85]{Chapter10B_03d.fig.pdf}\hspace*{0.5cm}}{0.48}
\end{Figure}
在\cref{fig:离散Gaussian卷积核}中,我们展示了离散化的$\sigma=1,r=3$的一维和二维的Gaussian卷积核。

\begin{Figure}[离散Gaussian卷积核]
    \figuresub[一维离散Gaussian卷积核]{\includegraphics[scale=0.8]{Chapter10B_05.fig.pdf}}{0.48}
    \figuresub[二维离散Gaussian卷积核]{\includegraphics[scale=0.8]{Chapter10B_06.fig.pdf}}{0.48}
\end{Figure}

通过这样的方式生成的二维卷积核称为“可分离的”。可分离的二维卷积核对于二维卷积计算而言,如\cref{fig:可分离的二维卷积核的优化计算}所示,若采用一定的优化算法,可以将计算复杂度从正比于$(2r+1)^2$降低至正比于$2(2r+1)$,即从$O(r^2)$降低到$O(r)$!这一提升对于半径较大的核而言是非常可观的。
\begin{Figure}[可分离的二维卷积核的优化计算]
    \figuresub[普通计算方法]{\includegraphics[scale=0.8]{Chapter10B_07.fig.pdf}}\vspace{12pt}

    \figuresub[优化计算方法(第一步)]{\includegraphics[scale=0.8]{Chapter10B_08.fig.pdf}}{0.48}
    \figuresub[优化计算方法(第二步)]{\includegraphics[scale=0.8]{Chapter10B_09.fig.pdf}}{0.48}
\end{Figure}

普通的二维卷积计算方法如下
\begin{Equation}
    c_2[i,j]=(a_2*b_2)[i,j]=\Sum[i'=i-r][i+r]\Sum[j'=j-r][j+r]a_2[i',j']b_2[i-i',j-j']
\end{Equation}
若$b_2$是可分离的,根据\cref{fml:可分离的二维卷积核}
\begin{Equation}
    c_2[i,j]=(a_2*b_2)[i,j]=\Sum[i'=i-r][i+r]\Sum[j'=j-r][j+r]a_2[i',j']b_1[i-i']b_1[j-j']
\end{Equation}
这里$b_1[i-i']$与$j'$无关,可以提出内层求和中
\begin{Equation}
    c_2[i,j]=(a_2*b_2)[i,j]=\Sum[i'=i-r][i+r]b_1[i-i']\Sum[j'=j-r][j+r]a_2[i',j']b_1[j-j']
\end{Equation}

不妨构造如下$s_2[i,j]$
\begin{Equation}
    s_2[i,j]=\Sum[j'=j-r][j+r]a_2[i,j']b_1[j-j']
\end{Equation}
此时$c_2[i,j]$就可以表示为
\begin{Equation}
    c_2[i,j]=\Sum[i'=i-r][i+r]s_2[i',j]b_1[i-i']
\end{Equation}
即当二维卷积核可分离时,可用两次$(2r+1)$的一维卷积替代一次$(2r+1)^2$的二维卷积!
