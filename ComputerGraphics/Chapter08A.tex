\section{光栅化直线}

在这一节,我们要研究的是如何光栅化一条直线,如\xref{fig:光栅化直线}所示。

\begin{Figure}[光栅化直线]
    \includegraphics[scale=0.8]{build/Chapter08A_01.fig}
\end{Figure}

直线的起点和终点用$\vb{p}_0$和$\vb{p}_1$表示
\begin{Equation}
    \vb{p}_0=(x_0,y_0)\qquad
    \vb{p}_1=(x_1,y_1)
\end{Equation}
直线方程可以表示为
\begin{Equation}
    f(x,y)=(y_0-y_1)x+(x_1-x_0)y+x_0y_1-x_1y_0=0
\end{Equation}
应指出,这里$(x_0,y_0)$和$(x_1,y_1)$未必是整数,换言之,起点和终点可以在屏幕上的任意位置。

第一步要做的是确定起点和终点临近的整数点
\begin{Gather}
    x_{\min}=\te{round}(x_0)\qquad y_{\min}=\te{round}(y_0)\\
    x_{\max}=\te{round}(x_1)\qquad y_{\max}=\te{round}(y_1)
\end{Gather}
我们需要根据直线$f(x,y)$的斜率$m$进行分类讨论,其中斜率$m$可以表示为
\begin{Equation}
    m=\frac{y_1-y_0}{x_1-x_0}
\end{Equation}

我们这里重点讨论$m\in(0,1]$,这种情况下直线在$x$方向比$y$方向增加的更快。由于我们期望得到的是一条最细且无空隙的直线,在$m\in(0,1]$的条件下,这将意味着两点:每一列只能存在一个像素,每一行至少存在一个像素。故我们将一直向右绘制像素,并在有些情况下向上抬高一格绘制像素。换言之,若已绘制像素$(x,y)$,下一像素只有$(x+1,y)$或$(x+1,y+1)$两种情况。判断的方法在于计算两者的中点$(x+1,y+0.5)$是否位于直线$f(x,y)=0$上方
\begin{itemize}
    \item 若$f(x+1,y+0.5)>0$,则中点位于直线上方,应有$(x,y)\to (x+1,y)$。
    \item 若$f(x+1,y+0.5)<0$,则中点位于直线下方,应有$(x,y)\to (x+1,y+1)$。
\end{itemize}

由此,我们就可以绘制出\xref{fig:光栅化直线}中的结果了。