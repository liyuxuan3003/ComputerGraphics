\section{着色频率}



在\xref{sec:着色方程}中,我们已经了解已知光源位置、观察原点、观察点的情况下如何计算观察到的颜色,现在的问题是,这怎么用来确定每一像素的颜色?不过首先要说明的一点是,尽管法向量通常是对于一个三角面而言的,但在着色时,我们常常会使用三角面“顶点的法向量”,这是如何定义的?如\xref{fig:顶点处的法向量},顶点处的法向量定义为共享该顶点的所有三角面的法向量的平均值。

\begin{Figure}[顶点处的法向量]
    \includegraphics[scale=0.8]{build/Chapter08D_01.fig.pdf}
\end{Figure}

在确定三角面对应像素的颜色时,有两种着色频率(Shading Frequency)
\begin{itemize}
    \item 逐顶点着色(Per-vertex Shading),如\xref{fig:逐顶点着色}所示。
    \item 逐片元着色(Per-fragment Shading),如\xref{fig:逐片元着色}所示。
\end{itemize}
\begin{Figure}[着色频率]
    \begin{FigureSub}[逐顶点着色]
        \includegraphics[scale=0.8]{build/Chapter08D_02.fig.pdf}
    \end{FigureSub}
    \hspace{0.5cm}
    \begin{FigureSub}[逐片元着色]
        \includegraphics[scale=0.8]{build/Chapter08D_03.fig.pdf}
    \end{FigureSub}
\end{Figure}
两者的差别在于,逐顶点着色是先根据顶点法向量$\vb{n}_0,\vb{n}_1,\vb{n}_2$通过\xref{sec:表面着色}介绍的着色方程计算顶点处的颜色$c_0,c_1,c_2$,随后通过插值$c=\alpha c_0+\beta c_1+\gamma c_2$得到三角面内某一特定像素的颜色,这一插值方法被称为重心坐标插值(Barycentric Interpolation)。在\xref{fig:光栅化三角形}中表现的就是逐顶点着色,顶点$\vb{p}_0,\vb{p}_1,\vb{p}_2$被分别假定为红、绿、蓝。然而,我们知道有一些着色方法对于法向量的法向量极为敏感,为此,逐片元着色先对法向量$\vb{n}_0,\vb{n}_1,\vb{n}_2$进行$\vb{n}=\alpha\vb{n}_0+\beta\vb{n}_1+\gamma\vb{n}_2$插值,这样一来三角面内就分布着均匀过渡的法向量,由此再计算颜色,就可以获得更均匀的着色效果。逐片元着色相较逐顶点着色具有更好的效果但也更为耗时,因此我们可以根据具体情况采用合适的着色频率,例如,Blinn-Phong着色对法向量非常敏感,Lambertian着色和Ambient着色对法向量则相对不敏感,两者就可以分别采用逐片元着色和逐顶点着色。

\begin{Figure}[着色频率的效果]
    \begin{FigureSub}[逐顶点着色的效果]
        \includegraphics[width=6.5cm]{build/SpherePerVertex.fig.pdf}
    \end{FigureSub}
    \hspace{0.5cm}
    \begin{FigureSub}[逐片元着色的效果]
        \includegraphics[width=6.5cm]{build/SpherePerFragment.fig.pdf}
    \end{FigureSub}
\end{Figure}

\xref{fig:着色频率的效果}展示了不同着色频率的效果差异,可以看出,逐顶点着色和逐片元着色的结果在大部分区域是没有差别的,然而,在对法向量变化较为敏感的高光区域,逐顶点着色的高光光斑表现处明显的三角面拼合图案,整体呈六边形。而相比之下,逐片元着色的高光光斑就自然很多。