\section{曲线的矩阵形式与混合函数}


先来考虑一个简单的问题,写出一个起点为$\vb{p}_0=\vb{f}(0)$和终点$\vb{p}_1=\vb{f}(1)$的一次曲线
\begin{Equation}
    \vb{f}(u)=(1-u)\vb{p}_0+u\vb{p}_1
\end{Equation}
但这类问题并不总是那么简单,考虑下面的二次曲线
\begin{itemize}
    \item 已知二次曲线的起点$\vb{p}_0=\vb{f}(0)$、起点的一阶导数$\vb{p}_1=\vb{f}'(0)$、终点$\vb{p}_2=\vb{f}(1)$,求方程。
    \item 已知二次曲线的起点$\vb{p}_0=\vb{f}(0)$、中点$\vb{p}_1=\vb{f}(0.5)$、终点$\vb{p}_2=\vb{f}(1)$,求方程。
\end{itemize}
显然,此时就很难通过观察直接写出曲线方程了。这类问题的抽象表述是:对于一条$d$次曲线$\vb{f}(u)=(x,y)$,需要总计$n=d+1$个控制点$\vb{p}_0,\vb{p_1},\cdots,\vb{p}_{n-1}$对曲线予以约束。控制点是通过曲线$\vb{f}(u)$在某一位置处的某一阶导数(及其线性组合)来定义。现在的问题是,对于给定的一系列控制点$\vb{p}_0,\vb{p}_1,\cdots,\vb{p}_{n-1}$以及控制点的约束方式,如何直接写出该曲线$\vb{f}(u)$的方程?

首先,对于一条$d$次曲线$\vb{f}(u)$,其标准形式为($d=n-1$)
\begin{Equation}
    \vb{f}(u)=\Sum[k=0][n-1]u^k\vb{a}_k=\vb{a}_0+u\vb{a}_1+u^2\vb{a}_2+\cdots+u^{n-1}\vb{a}_{n-1}
\end{Equation}
引入向量$\vb{u}$,它是多项式的各阶项
\begin{Equation}
    \vb{u}=(1,u,u^2,\cdots,u^{n-1})
\end{Equation}
引入向量$\vb{a}$,它是多项式的各阶系数(注意这里$\vb{a}$是“向量构成的向量”)
\begin{Equation}
    \vb{a}=(\vb{a}_0,\vb{a}_1,\vb{a}_2,\cdots,\vb{a}_{n-1})
\end{Equation}
这样一来,标准形式可以重写为
\begin{Equation}
    \vb{f}(u)=\vb{u}\vb{a}
\end{Equation}
整理如下
\begin{BoxFormula}[曲线的标准形式]
    曲线的标准形式为
    \begin{Equation}
        \vb{f}(u)=\vb{u}\vb{a}
    \end{Equation}
    亦可以展开写作
    \begin{Equation}
        \vb{f}(u)=\Sum[k=0][n-1]u^k\vb{a}_k
    \end{Equation}
\end{BoxFormula}

引入向量$\vb{p}$,它包含了所有的控制点
\begin{Equation}
    \vb{p}=(\vb{p}_0,\vb{p}_1,\vb{p}_2,\cdots,\vb{p}_{n-1})
\end{Equation}
由于控制点$\vb{p}_0,\vb{p}_1,\cdots,\vb{p}_{n-1}$均定义为$\vb{f}(u)$及其导数在某一位置的值,而$d$次曲线$\vb{f}(u)$可以先用\xref{fml:曲线的标准形式}的标准形式写出来,因此,根据控制点的定义,可以写出$\vb{p}$和$\vb{a}$间的关系式
\begin{Equation}
    \vb{p}=\vb{C}\vb{a}
\end{Equation}
这里矩阵$\vb{C}$称为约束矩阵(Constraint Matrix),它是控制点的约束方式的数学表达。
\begin{BoxFormula}[约束矩阵]
    约束矩阵$\vb{C}$描述了$\vb{p}$相对于$\vb{a}$的关系
    \begin{Equation}
        \vb{C}:\ \vb{p}=\vb{C}\vb{a}
    \end{Equation}
\end{BoxFormula}
假如需要反过来将$\vb{a}$用$\vb{p}$表示,则需要对$\vb{C}$求逆
\begin{Equation}
    \vb{a}=\vb{C}^{-1}\vb{p}=\vb{B}\vb{p}
\end{Equation}
这里矩阵$\vb{B}$称为偏置矩阵(Basis Matrix),它是约束矩阵的逆矩阵。
\begin{BoxFormula}[偏置矩阵]
    偏置矩阵$\vb{B}$描述了$\vb{a}$相对于$\vb{p}$的关系,偏置矩阵$\vb{B}$是约束矩阵$\vb{C}$的逆矩阵
    \begin{Equation}
        \vb{B}:\ \vb{a}=\vb{B}\vb{p}\qquad \vb{B}=\vb{C}^{-1}
    \end{Equation}
    % 偏置矩阵$\vb{B}$是约束矩阵$\vb{C}$的逆矩阵
    % \begin{Equation}
    %     \vb{B}=\vb{C}^{-1}
    % \end{Equation}
\end{BoxFormula}
在\xref{fml:曲线的标准形式}的标准形式$\vb{f}=\vb{u}\vb{a}$中代入\xref{fml:偏置矩阵}给出的$\vb{a}=\vb{B}\vb{p}$,即可得到
\begin{Equation}[]
    \vb{f}(u)=\vb{u}\vb{B}\vb{p}
\end{Equation}
至此,只要根据控制点的约束方式写出$\vb{C}$,对其求逆矩阵得到$\vb{B}=\vb{C}^{-1}$,我们就可以立即依据$\vb{f}(u)=\vb{u}\vb{B}\vb{p}$得到曲线的方程!这样一来,整个求解过程就被抽象为计算一个矩阵的逆。

除此之外,我们也可以从另一个角度理解上式
\begin{Equation}
    \vb{f}(u)=\Sum[k=0][n-1]b_k(u)\vb{p}_k\qquad \vb{b}(u)=\vb{u}\vb{B}
\end{Equation}
这就告诉我们,曲线上每一个点都可以视为曲线控制点的线性组合。当然,在不同的$u$处,控制点线性组合的比例是不同的,上式引入的$\vb{b}(u)$称为混合函数(Blending Function),其分量$b_k(u)$即代表控制点$\vb{p}_k$在曲线取值为$u$处的权重。混合函数很容易写出,例如$b_k(u)$就是关于$1,u,u^2,\cdots,u^{n-1}$的多项式,而各项前的系数则由偏置矩阵$\vb{B}$的第$k$列逐行给出。

\begin{BoxFormula}[混合函数]
    混合函数$\vb{b}(u)$定义为
    \begin{Equation}
        \vb{b}(u)=\vb{u}\vb{B}
    \end{Equation}
\end{BoxFormula}

\begin{BoxFormula}[曲线和控制点]
    曲线可以由其控制点表示为
    \begin{Equation}
        \vb{f}(u)=\vb{u}\vb{B}\vb{p}
    \end{Equation}
    亦可以用混合函数将曲线表达为控制点的线性组合
    \begin{Equation}
        \vb{f}(u)=\Sum[k=0][n-1]b_k(u)\vb{p}_k
    \end{Equation}
\end{BoxFormula}

接下里,让我们通过一次曲线和二次曲线的例子,实践一下上述求解方式。

\subsection{一次曲线}

\xref{fig:一次曲线}展示了一条一次曲线,控制点$\vb{p}_0,\vb{p}_1$代表了其起点和终点。
\begin{Figure}[一次曲线]
    \includegraphics[scale=0.8]{build/Chapter15B_02.fig.pdf}
\end{Figure}
一次曲线的通式是
\begin{Equation}
    \vb{f}(u)=\vb{a}_0+u\vb{a}_1
\end{Equation}
根据$\vb{p}_0,\vb{p}_1$的意义,可以写出
\begin{Gather}
    \vb{p}_0=\vb{f}(0)=\vb{a}_0\\
    \vb{p}_1=\vb{f}(1)=\vb{a}_0+\vb{a}_1
\end{Gather}
约束矩阵为
\begin{Equation}
    \vb{C}=
    \begin{pmatrix}
        1&0\\
        1&1\\
    \end{pmatrix}
\end{Equation}
偏置矩阵为
\begin{Equation}
    \vb{B}=\vb{C}^{-1}=
    \begin{pmatrix}
        1&0\\
        -1&1\\
    \end{pmatrix}
\end{Equation}
混合函数为
\begin{Gather}
    b_0(u)=1-u\\
    b_1(u)=u
\end{Gather}
故一次曲线的方程应为
\begin{Equation}
    \vb{f}(u)=(1-u)\vb{p}_0+u\vb{p}_1
\end{Equation}
将上述结论整理如下
\begin{BoxFormula}[一次曲线的约束矩阵]
    一次曲线的约束矩阵为
    \begin{Equation}
        \vb{C}=\begin{pmatrix}
            1&0\\
            1&1\\
        \end{pmatrix}
    \end{Equation}
\end{BoxFormula}

\begin{BoxFormula}[一次曲线的偏置矩阵]
    一次曲线的偏置矩阵为
    \begin{Equation}
        \vb{B}=\begin{pmatrix}
            1&0\\
            -1&1\\
        \end{pmatrix}
    \end{Equation}
\end{BoxFormula}

\begin{BoxFormula}[一次曲线的混合函数]
    一次曲线的混合函数为
    \begin{Gather}
        b_0(u)=1-u\\
        b_1(u)=u
    \end{Gather}
\end{BoxFormula}

\xref{fig:一次曲线的混合函数}展示了一次曲线的混合函数
\begin{Figure}[一次曲线的混合函数]
    \includegraphics[scale=0.8]{build/Chapter15B_01a.fig.pdf}
\end{Figure}

\subsection{二次曲线}
\xref{fig:二次曲线}展示了一条二次曲线,控制点$\vb{p}_0,\vb{p}_1$代表了其起点和终点。
\begin{Figure}[二次曲线]
    \includegraphics[scale=0.8]{build/Chapter15B_03.fig.pdf}
\end{Figure}
二次曲线的通式及其一阶导数为
\begin{Gather}
    \vb{f}(u)=\vb{a}_0+u\vb{a}_1+u^2\vb{a}_2\\
    \vb{f}'(u)=\vb{a}_1+2u\vb{a}_2
\end{Gather}
根据$\vb{p}_0,\vb{p}_1,\vb{p}_2$的意义,可以写出
\begin{Gather}
    \vb{p}_0=\vb{f}(0)=\vb{a}_0\\
    \vb{p}_1=\vb{f}'(0)=\vb{a}_1\\
    \vb{p}_2=\vb{f}(1)=\vb{a}_0+\vb{a}_1+\vb{a}_2
\end{Gather}
约束矩阵为
\begin{Equation}
    \vb{C}=\begin{pmatrix}
        1&0&0\\
        0&1&0\\
        1&1&1\\
    \end{pmatrix}
\end{Equation}
偏置矩阵为
\begin{Equation}
    \vb{B}=\vb{C}^{-1}=\begin{pmatrix}
        1&0&0\\
        0&1&0\\
        -1&-1&1\\
    \end{pmatrix}
\end{Equation}
混合函数为
\begin{Gather}
    b_0(u)=1-u^2\\
    b_1(u)=u-u^2\\
    b_2(u)=u^2
\end{Gather}
故二次曲线的方程应为
\begin{Equation}
    \vb{f}(u)=(1-u)^2\vb{p}_0+(u-u^2)\vb{p}_1+u^2\vb{p}_2
\end{Equation}
将上述结论整理如下
\begin{BoxFormula}[二次曲线的约束矩阵]
    二次曲线的约束矩阵为
    \begin{Equation}
        \vb{C}=\begin{pmatrix}
            1&0&0\\
            0&1&0\\
            1&1&1\\
        \end{pmatrix}
    \end{Equation}
\end{BoxFormula}

\begin{BoxFormula}[二次曲线的偏置矩阵]
    二次曲线的偏置矩阵为
    \begin{Equation}
        \vb{B}=\begin{pmatrix}
            1&0&0\\
            0&1&0\\
            -1&-1&1\\
        \end{pmatrix}
    \end{Equation}
\end{BoxFormula}

\begin{BoxFormula}[二次曲线的混合函数]
    二次曲线的混合函数为
    \begin{Gather}
        b_0(u)=1-u^2\\
        b_1(u)=u-u^2\\
        b_2(u)=u^2
    \end{Gather}
\end{BoxFormula}

\xref{fig:二次曲线的混合函数}展示了二次曲线的混合函数
\begin{Figure}[二次曲线的混合函数]
    \includegraphics[scale=0.8]{build/Chapter15B_01b.fig.pdf}
\end{Figure}
至此,我们可以看出,这套“约束矩阵、偏置矩阵、混合函数”的计算方式,可以普适解决根据曲线控制点的约束条件求解曲线方程的问题,且为约束条件提供了一套简洁的表征方式。

