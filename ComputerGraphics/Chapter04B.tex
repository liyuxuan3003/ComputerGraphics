\section{光线产生}
光线产生和投影方式有关,主要有以下两种投影
\begin{itemize}
    \item 平行投影(Parallel Projection),如\xref{fig:平行投影}所示,是指光线从每个像素的中心处发出且所有光线是平行的。平行投影可以分为两种情况,取决于光线方向是否与像平面垂直,分别称为正交投影(Orthographic)和斜交投影(Oblique),其中正交投影比较常见,我们在\xref{fig:平行投影}展示的也是正交投影的情况。平行投影常用于工程图(例如机械和建筑)的绘制,它的最大特征是,三维空间中的平行线经过投影后在二维图像中仍然是平行的。
    \item 透视投影(Perspective Projection),如\xref{fig:透视投影}所示,是指光线从一个特定的观察点处向各个像素的方向发出,观察点位于像平面中心后方处。透视投影符合人眼或相机的成像模式,它的特点是具有透视效应,近大远小,平行线不一定再平行。试想,站在一条笔直的公路中央望向远方,会发现公路两侧边沿会逐渐靠拢,但它们实际是平行的。
\end{itemize}

\begin{Figure}[两种投影方式]
    \begin{FigureSub}[平行投影]
        \includegraphics[scale=0.8]{build/Chapter04B_01.fig.pdf}
    \end{FigureSub}
    \begin{FigureSub}[透视投影]
        \includegraphics[scale=0.8]{build/Chapter04B_02.fig.pdf}
    \end{FigureSub}
\end{Figure}

有趣的是,平行投影在一定意义上可以认为是观察点位于无穷远处的透视投影。这一理解最直观的感受是,长焦镜头具有压缩空间的效果,即透视带来的近大远小在长焦下不太明显了。

在\xref{fig:两种投影方式}中,有一些标注需要说明。首先,无论是平行投影还是透视投影,我们都会确定一个原点$\vb{e}$,对于平行投影是像平面中心,对于透视投影是观察点。其次,在$\vb{e}$上标注了三个重要的单位矢量$\vb{u},\vb{v},\vb{w}$,三者呈右手螺旋关系。其中,$\vb{w}$是视野方向的反方向,$\vb{v},\vb{u}$均位于像平面,它们会定义画面的“上”和“右”在空间中的对应方向。最后,$u_l,u_r,v_b,v_t$分别确定了画面在左、右、下、上的边界。特别的,在透视投影中,$d$还用于表示观察点和像平面间的距离。

假设二维图像的像素总数是$n_x\times n_y$,对于$(i,j)$处的像素,其水平位置$u$和垂直位置$v$为
\begin{Gather}
    u=u_l+(u_r-u_l)(i+0.5)/n_x\\
    v=v_b+(v_t-v_b)(j+0.5)/n_y
\end{Gather}

我们可以用一个参数方程表达三维空间的直线(即这里的光线)
\begin{Equation}[光线方程]
    \vb{p}(t)=\vb{e}+\vb{d}t
\end{Equation}
其中,$\vb{e}$代表光线的原点,$\vb{d}$代表光线的方向。应指出的是光线的原点$\vb{e}$未必是\xref{fig:两种投影方式}中投影的原点$\vb{e}$,为强调这一区别,在本小节将光线方程$\vb{p}(t)=\vb{e}+\vb{d}t$中的$\vb{e},\vb{d}$记为$\vb{e}_{ray},\vb{d}_{ray}$。

对于平行投影,光线的原点在变化,但方向不变
\begin{BoxFormula}[平行投影的光线方程]
    平行投影的光线方程,原点$\vb{e}_{ray}$和方向$\vb{d}_{ray}$分别为
    \begin{Gather}
        \vb{e}_{ray}=\vb{e}+u\vb{u}+v\vb{v}\\
        \vb{d}_{ray}=-\vb{w}
    \end{Gather}
\end{BoxFormula}

对于透视投影,光线的方向在变化,但原点不变
\begin{BoxFormula}[透视投影的光线方程]
    透视投影的光线方程,原点$\vb{e}_{ray}$和方向$\vb{d}_{ray}$分别为
    \begin{Gather}
        \vb{e}_{ray}=\vb{e}\\
        \vb{d}_{ray}=-\vb{d}\vb{w}+u\vb{u}+v\vb{v}
    \end{Gather}
\end{BoxFormula}