\chapter{矩阵变换}
渲染分为图像顺序渲染和物体顺序渲染两种,在\xref{chap:光线追踪}中,我们已经介绍了图像顺序渲染的典型算法光线追踪。在接下来的几章,我们将转向物体顺序渲染,主要分为两个部分
\begin{itemize}
    \item 视角变换(Viewing Transformation),研究如何将三维空间的坐标变化到二维平面上。
    \item 光栅化(Raterization),研究如何将二维平面上的图形转换为离散的像素点。
\end{itemize}
而本章,我们将研究视角变换部分的基础数学,即如何用矩阵实现各种变换?首先我们将从二维出发介绍四种基本的矩阵变换,随后我们会介绍矩阵的特征值分解和奇异值分解,进而将矩阵变换推广至三维并解决三维中常用的法向量在矩阵变换中遇到的特殊问题,最后我们会研究如何通过齐次坐标优雅的表示仿射变换,并以此为基础,介绍坐标系变换的表示方法。

\section{矩阵变换}

