\chapter{线性代数}
% 渲染分为图像顺序渲染和物体顺序渲染两种,在\cref{chap:光线追踪}中,我们已经介绍了图像顺序渲染的典型算法光线追踪。在接下来的几章,我们将转向物体顺序渲染,主要分为两个部分
% \begin{itemize}
%     \item 视角变换(Viewing Transformation),研究如何将三维空间的坐标变化到二维平面上。
%     \item 光栅化(Raterization),研究如何将二维平面上的图形转换为离散的像素点。
% \end{itemize}

在本章,我们将研究视角变换部分的基础数学,即如何用矩阵实现各种变换?首先我们将从二维出发介绍四种基本的矩阵变换,随后我们会介绍矩阵的特征值分解和奇异值分解,进而将矩阵变换推广至三维并解决三维中常用的法向量在矩阵变换中遇到的特殊问题,最后我们会研究如何通过齐次坐标优雅的表示仿射变换,并以此为基础,介绍坐标系变换的表示方法。

\section{矩阵变换}


\section{矩阵变换的分解}
本节将研究这样一个问题,对于一般的变换矩阵$\vb{A}$,如何将其拆分为若干基本变换的叠加?

在开始前,有必要说明一下几种特殊矩阵的定义。
\begin{BoxDefinition}[对角矩阵]
    对角矩阵(Diagonal Matrix)是指仅具有对角元素的矩阵
    \begin{Equation}
        \vb{A}=\te{diag}(a_1,a_2)=\begin{pmatrix}
            a_1&0\\
            0&a_2\\
        \end{pmatrix}
    \end{Equation}
\end{BoxDefinition}

\begin{BoxDefinition}[对称矩阵]
    对称矩阵(Symmetric Matrix)是指转置与自身相等的矩阵
    \begin{Equation}
        \vb{A}=\vb{A}^T
    \end{Equation}
\end{BoxDefinition}
显然,对角矩阵一定是一个对称矩阵。

\begin{BoxDefinition}[正交矩阵]
    正交矩阵(Orthogonal Matrix)是指转置与自身的积为单位阵的矩阵
    \begin{Equation}
        \vb{A}^T\vb{A}=\vb{I}
    \end{Equation}
\end{BoxDefinition}
这里有一个很重要的性质:若一个矩阵$\vb{A}$的列是由一组相互正交的单位向量组成,那么该矩阵一定是一个正交矩阵!试想,对于$\vb{A}^T\vb{A}$的乘积结果,其对角元素是一个单位向量与其自身的点积,因而是$1$,其非对角元素是两个正交向量的点积,因而是$0$,这样就有$\vb{A}^T\vb{A}=\vb{I}$。

\subsection{特征值分解}
在继续介绍特征值分解前,我们有必要先回顾一下特征值和特征向量的概念。

对于一个矩阵$\vb{A}$,若能找到向量$\vb{a}$和数$\lambda$使
\begin{Equation}
    \vb{A}\vb{a}=\lambda\vb{a}
\end{Equation}
那么$\vb{a},\lambda$就称为$\vb{A}$的一组特征向量(Eigen Vector)和特征值(Eigen Value)。换言之,若矩阵作用在一个向量上却只会令这个向量伸长或压缩,那么这个向量就是矩阵的特征向量。

很重要的问题是如何寻找特征向量和特征值?我们可以先加入一个$\vb{I}$\setpeq{求解特征向量和特征值}
\begin{Equation}&[1]
    \vb{A}\vb{a}=\lambda\vb{I}\vb{a}
\end{Equation}
移项至左侧
\begin{Equation}&[2]
    \vb{A}\vb{a}-\lambda\vb{I}\vb{a}=0
\end{Equation}
提出矩阵
\begin{Equation}&[3]
    (\vb{A}-\lambda\vb{I})\vb{a}=0
\end{Equation}
该方程成立唯一的条件是$\vb{A}-\lambda\vb{I}$是一个奇异阵,换言之,行列式为$0$。假设$\vb{A}$是$2\times 2$的
\begin{Equation}&[4]
    \begin{vmatrix}
        a_{11}-\lambda&a_{12}\\
        a_{21}&a_{22}-\lambda\\
    \end{vmatrix}=0
\end{Equation}
即
\begin{Equation}&[5]
    \lambda^2-(a_{11}+a_{22})\lambda+(a_{11}a_{22}-a_{12}a_{21})=0
\end{Equation}
这是一个二次方程,故可以解出两个特征值$\lambda$。事实上,对于一个$n\times n$的矩阵,其具有$n$个特征值。由于四次以上的方程无系解析解,故对于$n>4$的矩阵,其特征值只有数值解。


而对于某个已经求出的特征值$\lambda$,也很容易确定对应的特征向量$\vb{a}=(x,y)$,考虑\xrefpeq{3}
\begin{Equation}
    \begin{pmatrix}
        a_{11}-\lambda&a_{12}\\
        a_{21}&a_{22}-\lambda\\
    \end{pmatrix}
    \begin{pmatrix}
        x\\
        y\\
    \end{pmatrix}=
    \begin{pmatrix}
        0\\
        0\\
    \end{pmatrix}
\end{Equation}
这是一个齐次方程,有无穷多组非零$(x,y)$是方程的解,注意到
\begin{Equation}
    (a_{11}-\lambda)x+a_{12}y=a_{21}x+(a_{22}-\lambda)y
\end{Equation}
整理得到
\begin{Equation}
    (a_{11}-a_{21}-\lambda)x=(a_{22}-a_{12}-\lambda)y
\end{Equation}
因此,只要$x,y$符合下面的比例,那它就是$\lambda$对应的特征向量
\begin{Equation}
    \frac{y}{x}=\frac{a_{11}-a_{21}-\lambda}{a_{22}-a_{12}-\lambda}
\end{Equation}
这一点从特征值的定义$\vb{A}\vb{a}=\lambda \vb{a}$上看也很合理:若$\vb{a}$是特征向量,那么$\vb{a}$的任意倍也都是特征向量,毕竟这只不过是在方程两端同乘了一个相同的系数罢了。但通常,我们会选择那个长度为$1$的那个单位特征向量。除此之外,可以证明一个矩阵的特征向量是两两相互正交的。\goodbreak

任何一个矩阵都具有特征向量和特征值,然而,若矩阵是满足对称矩阵,则矩阵自身可以直接用其特征向量和特征值表示出来,这就是所谓的特征值分解!它的具体形式如下
\begin{BoxFormula}[特征值分解]
    特征值分解(Eigen Value Decomposition, EVD)是指,对于对称矩阵$\vb{A}$,有
    \begin{Equation}
        \vb{A}=\vb{R}\vb{S}\vb{R}^T
    \end{Equation}
    矩阵$\vb{R}$是一个正交矩阵,由特征向量$\vb{r}_1,\vb{r}_2$构成
    \begin{Equation}
        \vb{R}=\qty(\vb{r}_1,\vb{r}_2)
    \end{Equation}
    矩阵$\vb{S}$是一个对角矩阵,由特征值$\lambda_1,\lambda_2$构成
    \begin{Equation}
        \vb{S}=\te{diag}(\lambda_1,\lambda_2)
    \end{Equation}
\end{BoxFormula}

现在我们来解读结果,不过再次之前,我们先要将特殊矩阵和矩阵变换联系起来
\begin{itemize}
    \item 对角矩阵$\vb{S}=\te{diag}(\lambda_1,\lambda_2)$对应缩放变换,其中$\lambda_1,\lambda_2$代表沿$\vb{x},\vb{y}$的缩放比。
    \item 正交矩阵$\vb{R}=(\vb{r}_1,\vb{r}_2)$对应旋转变换,其代表$\vb{x},\vb{y}$旋转至$\vb{r}_1,\vb{r}_2$方向,$\vb{R}^T$则相反。
\end{itemize}

特征值分解可以视为这样三步变换
\begin{enumerate}
    \item 旋转,矩阵$\vb{R}^T$作用,代表图形由$\vb{r}_1,\vb{r}_2$方向分别旋转至$\vb{x},\vb{y}$方向。
    \item 缩放,矩阵\hspace{0.45em}$\vb{S}$\hspace{0.45em}作用,代表图形沿$\vb{x},\vb{y}$方向分别缩放$\lambda_1,\lambda_2$倍。
    \item 旋转,矩阵$\vb{R}$作用,代表图形由$\vb{x},\vb{y}$方向分别旋转至$\vb{r}_1,\vb{r}_2$方向。
\end{enumerate}
特征值分解告诉我们,任何对称矩阵代表的变换,本质上都可以视为由“旋转--缩放--旋转”这样三步基本变换的组合变换。更进一步的说,由于缩放沿$\vb{x},\vb{y}$进行,且缩放前后的旋转分别是从特征向量方向旋转至$\vb{x},\vb{y}$再旋转回去,因此,对称矩阵的变换的效果其实就是在特征向量$\vb{r}_1,\vb{r}_2$的方向上分别进行特征值$\lambda_1,\lambda_2$倍的缩放!这是一个非常深刻且直观的几何理解。

\subsection{奇异值分解}
奇异值和奇异值分解的想法是从特征值分解延拓出来的。特征值分解存在一个明显的缺陷,即只有对角矩阵才能适用。奇异值分解期望将任何矩阵都拆分为“旋转--缩放--旋转”的形式,不过势必要做出一些让步,需要将$\vb{A}=\vb{R}\vb{S}\vb{R}^T$需改写为$\vb{A}=\vb{U}\vb{S}\vb{V}^T$,即首尾两个正交矩阵不再是同一个。将左侧$\vb{U}=(\vb{u}_1,\vb{u}_2)$和右侧$\vb{v}=(\vb{v}_1,\vb{v}_2)$中的向量分别称为左奇异向量和右奇异向量(Singular Vector),同时,将$\vb{S}=\te{diag}(\sigma_1,\sigma_2)$中的数称为奇异值(Singular Value)。

\begin{BoxFormula}[奇异值分解]
    奇异值分解(Singular Value Decomposition, SVD)是指,对于矩阵$\vb{A}$,有
    \begin{Equation}
        \vb{A}=\vb{U}\vb{S}\vb{V}^T
    \end{Equation}
    矩阵$\vb{U},\vb{V}$是正交矩阵,由左奇异向量$\vb{u}_1,\vb{u}_2$和右奇异向量$\vb{v}_1,\vb{v}_2$构成
    \begin{Equation}
        \vb{U}=\qty(\vb{u}_1,\vb{u}_2)\qquad
        \vb{V}=\qty(\vb{v}_1,\vb{v}_2)
    \end{Equation}
    矩阵$\vb{S}$是一个对角矩阵,由特征值$\sigma_1,\sigma_2$构成
    \begin{Equation}
        \vb{S}=\te{diag}(\sigma_1,\sigma_2)
    \end{Equation}
\end{BoxFormula}

奇异值分解可以视为这样三步变换
\begin{itemize}
    \item 旋转,矩阵$\vb{V}^T$作用,代表图形由$\vb{v}_1,\vb{v}_2$方向分别旋转至$\vb{x},\vb{y}$方向。
    \item 缩放,矩阵\hspace{0.45em}$\vb{S}$\hspace{0.45em}作用,代表图形沿$\vb{x},\vb{y}$方向分别缩放$\sigma_1,\sigma_2$倍。
    \item 旋转,矩阵$\vb{U}$作用,代表图形由$\vb{x},\vb{y}$方向分别旋转至$\vb{u}_1,\vb{u}_2$方向。
\end{itemize}

奇异值分解和特征值分解能如何联系在一起呢?考虑以下矩阵
\begin{Equation}
    \vb{M}=\vb{A}\vb{A}^T
\end{Equation}
代入$\vb{A}=\vb{U}\vb{S}\vb{V}^T$
\begin{Equation}
    \vb{M}=(\vb{U}\vb{S}\vb{V}^T)(\vb{U}\vb{S}\vb{V}^T)^T
\end{Equation}
若干矩阵乘积的转置,等于每个矩阵转置的逆序乘积
\begin{Equation}
    \vb{M}=(\vb{U}\vb{S}\vb{V}^T)(\vb{V}\vb{S}^T\vb{U}^T)
\end{Equation}
由于$\vb{S}$是对角矩阵,故$\vb{S}^T=\vb{S}$
\begin{Equation}
    \vb{M}=(\vb{U}\vb{S}\vb{V}^T)(\vb{V}\vb{S}\vb{U}^T)
\end{Equation}
应用矩阵乘法的分配律
\begin{Equation}
    \vb{M}=\vb{U}\vb{S}(\vb{V}^T\vb{V})\vb{S}\vb{U}^T
\end{Equation}
由于$\vb{V}$是对角矩阵,故$\vb{V}^T\vb{V}=\vb{I}$
\begin{Equation}
    \vb{M}=\vb{U}\vb{S}^2\vb{U}^T
\end{Equation}
这意味着$\vb{M}=\vb{A}^T\vb{A}$必然可以做特征值分解!且此时,特征向量是$\vb{A}$的左奇异向量,特征值是$\vb{A}$的奇异值的平方,因为$\vb{S}^2=\te{diag}(\sigma_1^2,\sigma_2^2)$。由此可见,通过这种方式,奇异值的计算可以转换为特征值的计算。并且,$\vb{A}^T\vb{A}=\vb{U}\vb{S}^2\vb{U}^T$给出$\vb{U}$,类似的,$\vb{A}\vb{A}^T=\vb{V}\vb{S}^2\vb{V}^T$给出$\vb{V}$。
\section{三维中的矩阵变换}
三维中的矩阵变换和二维中的情形类似,我们将相关矩阵列举如下。
\begin{BoxDefinition}[三维缩放变换]
    三维缩放变换定义为
    \begin{Equation}
        \te{scale}(s_x,s_y,s_z)=
        \begin{pmatrix}
            s_x&0&0\\
            0&s_y&0\\
            0&0&s_z\\
        \end{pmatrix}
    \end{Equation}
\end{BoxDefinition}

三维中的剪切有沿$x,y,z$三个方向的,每一种剪切都涉及另外两个方向的剪切系数。
\begin{BoxDefinition}[三维剪切变换]
    三维中,沿$x$方向的剪切变换定义为
    \begin{Equation}
        \te{shear}_x(d_y,d_z)=
        \begin{pmatrix}
            1&d_y&d_z\\
            0&1&0\\
            0&0&1\\
        \end{pmatrix}
    \end{Equation}
    三维中,沿$y$方向的剪切变换定义为
    \begin{Equation}
        \te{shear}_y(d_z,d_x)=
        \begin{pmatrix}
            1&0&0\\
            d_x&1&d_z\\
            0&0&1\\
        \end{pmatrix}
    \end{Equation}
    三维中,沿$z$方向的剪切变换定义为
    \begin{Equation}
        \te{shear}_z(d_x,d_y)=
        \begin{pmatrix}
            1&0&0\\
            0&1&0\\
            d_x&d_y&1\\
        \end{pmatrix}
    \end{Equation}
\end{BoxDefinition}

三维中的旋转有一些不一样。我们知道,旋转是围绕一根轴在一个面内进行的,二维的旋转只能在二维平面上进行,故只有一种旋转矩阵,但在三维下旋转则可以绕$x,y,z$轴旋转。
\begin{BoxDefinition}[三维旋转变换]*
    三维中,绕$x$轴的旋转变换定义为
    \begin{Equation}
        \te{rotate}_x(\phi)=
        \begin{pmatrix}
            1&0&0\\
            0&\cos\phi&-\sin\phi\\
            0&\sin\phi&\cos\phi\\
        \end{pmatrix}
    \end{Equation}
    三维中,绕$y$轴的旋转变换定义为
    \begin{Equation}
        \te{rotate}_y(\phi)=
        \begin{pmatrix}
            \cos\phi&0&\sin\phi\\
            0&1&0\\
            -\sin\phi&0&\cos\phi\\
        \end{pmatrix}
    \end{Equation}
    三维中,绕$z$轴的旋转变换定义为
    \begin{Equation}
        \te{rotate}_z(\phi)=
        \begin{pmatrix}
            \cos\phi&-\sin\phi&0\\
            \sin\phi&\cos\phi&0\\
            0&0&1\\
        \end{pmatrix}
    \end{Equation}
\end{BoxDefinition}

三维中的反射有翻转$x,y,z$轴三种,但下标记录的是以什么面进行对称。

\begin{BoxDefinition}[三维翻转变换]*
    三维中,以$y,z$平面为对称面(翻转$x$轴)的反射变换定义为
    \begin{Equation}
        \te{reflect}_{yz}=
        \begin{pmatrix}
            -1&0&0\\
            0&1&0\\
            0&0&1\\
        \end{pmatrix}
    \end{Equation}
    三维中,以$z,x$平面为对称面(翻转$y$轴)的反射变换定义为
    \begin{Equation}
        \te{reflect}_{zx}=
        \begin{pmatrix}
            1&0&0\\
            0&-1&0\\
            0&0&1\\
        \end{pmatrix}
    \end{Equation}
    三维中,以$x,y$平面为对称面(翻转$z$轴)的反射变换定义为
    \begin{Equation}
        \te{reflect}_{xy}=
        \begin{pmatrix}
            1&0&0\\
            0&1&0\\
            0&0&-1\\
        \end{pmatrix}
    \end{Equation}
\end{BoxDefinition}

\subsection{法向量变换}

在三维空间中,我们经常会用法向量表示一个面的方向,然而,法向量进行变换后往往不再是法向量。\cref{fig:法向量在三维变换中的问题}展示了这一问题,矩形经过变换矩阵$\vb{M}$的剪切变换后,法向量$\vb{n}$顺着剪切变为了斜向上的$\vb{M}\vb{n}$,但其实$\vb{M}\vb{n}$并不垂直于平面,此时,正确的法向量应是图中标出的$\vb{N}\vb{n}$。

这就告诉我们,当图形通过矩阵$\vb{M}$变换时,法向量若也通过矩阵$\vb{M}$变换将得到不正确的结果,法向量需要另外一个不同的矩阵$\vb{N}$矩形进行变换。故现在的问题是怎么由$\vb{M}$求出$\vb{N}$?

\begin{Figure}[法向量在三维变换中的问题]
    \figuresub[变换前;变换前--法向量在三维变换中的问题]{\includegraphics[scale=0.8]{Chapter06C_01.fig.pdf}}{0.48}
    \figuresub[变换后;变换后--法向量在三维变换中的问题]{\includegraphics[scale=0.8]{Chapter06C_02.fig.pdf}}{0.48}
\end{Figure}
首先,可以确定的是变换前法向量$\vb{n}$一定垂直于切向量$\vb{t}$
\begin{Equation}[法向量变换1]
    \vb{n}^T\vb{t}=0
\end{Equation}

我们将变换后的向量用$\vb{n}_N=\vb{N}\vb{n},~\vb{n}_M=\vb{M}\vb{n},~\vb{t}_M=\vb{M}\vb{t}$表示,我们要找到$\vb{N}$使下式成立
\begin{Equation}[法向量变换2]
    \vb{n}_N^T\vb{t}_M=0
\end{Equation}
我们可以在\cref{eq:法向量变换1}做一些变形,利用$\vb{M}^{-1}\vb{M}=\vb{I}$
\begin{Equation}
    \vb{n}^T\vb{t}=\vb{n}^T\vb{I}\vb{t}=\vb{n}^T\vb{M}^{-1}\vb{M}\vb{t}=0
\end{Equation}
应用矩阵乘法的分配律
\begin{Equation}
    (\vb{n}^T\vb{M}^{-1})(\vb{M}\vb{t})=0
\end{Equation}
注意到$\vb{M}\vb{t}$就是$\vb{t}_M$
\begin{Equation}
    (\vb{n}^T\vb{M}^{-1})\vb{t}_M=0
\end{Equation}
比照\cref{eq:法向量变换2},即得
\begin{Equation}
    \vb{n}_N^T=\vb{n}^T\vb{M}^{-1}
\end{Equation}
两边转置
\begin{Equation}
    \vb{n}_N=(\vb{M}^{-1})^T\vb{n}
\end{Equation}

由此可见应有$\vb{N}=(\vb{M}^{-1})^T$,即适用法向量变换的$\vb{N}$是$\vb{M}$的逆矩阵的转置。
\begin{BoxFormula}[法向量变换]
    法向量的变换矩阵$\vb{N}$为
    \begin{Equation}
        \vb{N}=(\vb{M}^{-1})^T
    \end{Equation}
\end{BoxFormula}

我们可以证明,这里$\vb{N}$可以写为
\begin{Equation}
    \vb{N}=
    \begin{pmatrix}
        m_{11}^c&m_{12}^c&m_{13}^c\\
        m_{21}^c&m_{22}^c&m_{23}^c\\
        m_{31}^c&m_{32}^c&m_{33}^c\\    
    \end{pmatrix}
\end{Equation}
其中$m_{ij}^c$代表$\vb{M}$在$m_{ij}$处的代数余子式,完整展开的结果是
\begin{Equation}
    \vb{N}=
    \begin{pmatrix}
        m_{22}m_{33}-m_{23}m_{32}&
        m_{23}m_{31}-m_{21}m_{33}&
        m_{21}m_{32}-m_{22}m_{31}\\
        m_{13}m_{32}-m_{12}m_{33}&
        m_{11}m_{33}-m_{13}m_{31}&
        m_{12}m_{31}-m_{11}m_{32}\\
        m_{12}m_{23}-m_{13}m_{22}&
        m_{13}m_{21}-m_{11}m_{23}&
        m_{11}m_{22}-m_{12}m_{21}\\
    \end{pmatrix}
\end{Equation}
实际上,严格来说上式还要再除以$\vb{M}$的行列式才是真正的$\vb{N}=(\vb{M}^{-1})^T$,但由于法向量并不关心长度(反正最终都要对$\vb{N}\vb{n}$归一化),计算$\vb{N}=(\vb{M}^{-1})^T$时省略除行列式不影响结果。
\section{仿射变换与平移}
通过前面几节,我们已经很清楚的认识到,在三维空间中的一个线性变换可以写成
\begin{Equation}
    \begin{pmatrix}
        x'\\
        y'\\
        z'\\
    \end{pmatrix}=
    \begin{pmatrix}
        m_{11}&m_{12}&m_{13}\\
        m_{21}&m_{22}&m_{23}\\
        m_{31}&m_{32}&m_{33}
    \end{pmatrix}
    \begin{pmatrix}
        x\\
        y\\
        z\\
    \end{pmatrix}
\end{Equation}
我们可以展开来写
\begin{Gather}
    x'=m_{11}x+m_{12}y+m_{13}z\\
    y'=m_{21}x+m_{22}y+m_{23}z\\
    z'=m_{31}x+m_{32}y+m_{33}z
\end{Gather}
我们通过不同的系数可以用上式表示缩放、剪切、旋转、反射等各种变换,但是,却无法表示最简单的平移!换言之,所有变换都是以$(0,0,0)$为中心进行的。若要考虑平移,则需有
\begin{Gather}
    x'=m_{11}x+m_{12}y+m_{13}z+x_t\\
    y'=m_{21}x+m_{22}y+m_{23}z+y_t\\
    z'=m_{31}x+m_{32}y+m_{33}z+z_t
\end{Gather}
诚然,我们可以将$\vb{r}'=\vb{M}\vb{r}$改写为$\vb{r}'=\vb{M}\vb{r}+\vb{r}_t$来单独处理平移的影响。然而这样做不够优雅,我们希望将平移也囊括在矩阵乘法内,这该怎么做呢?有一个极为巧妙的方法是引入齐次坐标(Homogeneous Coordinates)的概念,简单来说,它为每个向量多加了一个维度
\begin{Equation}
    \begin{pmatrix}
        x\\
        y\\
        z\\
        1\\
    \end{pmatrix}\qquad
    \begin{pmatrix}
        x\\
        y\\
        z\\
        0\\
    \end{pmatrix}
\end{Equation}
这个新增的第四维度的元素可以被填充为$0$和$1$
\begin{itemize}
    \item 添加$1$,代表该向量是一个“位矢”,(绝对位置、方位)。
    \item 添加$0$,代表该向量是一个“位移”,(相对位置、方向)。
\end{itemize}
这个设计是很合理且自然的,试想,两个“位置”的差就自然是一个“位矢”
\begin{Equation}
    \begin{pmatrix}
        x_1\\
        y_1\\
        z_1\\
        1\\
    \end{pmatrix}-
    \begin{pmatrix}
        x_2\\
        y_2\\
        z_2\\
        1\\
    \end{pmatrix}=
    \begin{pmatrix}
        x_1-x_2\\
        y_1-y_2\\
        z_1-z_2\\
        0\\
    \end{pmatrix}
\end{Equation}
而有了齐次坐标的定义后,平移就可以通过矩阵乘一并处理了
\begin{Equation}
    \begin{pmatrix}
        x'\\
        y'\\
        z'\\
        1\\
    \end{pmatrix}=
    \begin{pmatrix}
        m_{11}&m_{12}&m_{13}&x_t\\
        m_{21}&m_{22}&m_{23}&y_t\\
        m_{31}&m_{32}&m_{33}&z_t\\
        0&0&0&1\\
    \end{pmatrix}
    \begin{pmatrix}
        x\\
        y\\
        z\\
        1\\
    \end{pmatrix}
\end{Equation}

要注意,这里是“\textbf{先做线性变换,后做平移变换}”,若拆分开来写,则先作用的在后
\begin{Equation}
    \begin{pmatrix}
        x'\\
        y'\\
        z'\\
        1\\
    \end{pmatrix}=
    \begin{pmatrix}
        1&0&0&x_t\\
        0&1&0&y_t\\
        0&0&1&z_t\\
        0&0&0&1\\
    \end{pmatrix}
    \begin{pmatrix}
        m_{11}&m_{12}&m_{13}&0\\
        m_{21}&m_{22}&m_{23}&0\\
        m_{31}&m_{32}&m_{33}&0\\
        0&0&0&1\\
    \end{pmatrix}
    \begin{pmatrix}
        x\\
        y\\
        z\\
        1\\
    \end{pmatrix}
\end{Equation}

这种包含了平移变换的线性变换称为仿射变换(Affine Transformation)。这里可以看出齐次坐标最妙的一点,它不仅解决了平移,而且,倘若参与仿射变换的齐次坐标中的第四维度元素是$0$而不是$1$,即代表参与变换的是一个“位移”而不是“位矢”,则$x_t,y_t,z_t$代表的平移不会作用,且变换后该位置仍为$0$。这是合理的,因为相对距离不会在整体平移中发生变化!

\begin{BoxDefinition}[仿射变换]
    仿射变换矩阵为
    \begin{Equation}
        \vb{M}=
        \begin{pmatrix}
            m_{11}&m_{12}&m_{13}&x_t\\
            m_{21}&m_{22}&m_{23}&y_t\\
            m_{31}&m_{32}&m_{33}&z_t\\
            0&0&0&1\\
        \end{pmatrix}
    \end{Equation}
\end{BoxDefinition}

仿射变换和齐次坐标极大的简化了变换的表示,之后我们还会进一步拓展齐次坐标的用途。

\section{坐标系变换}
在这一节,我们讨论有关坐标系变换的问题,如\cref{fig:坐标系变换}所示。我们知道,在三维空间下的坐标系可以由其原点和三个基矢表征,$xyz$坐标系包含$\vb{o},\vb{x},\vb{y},\vb{z}$,$uvw$坐标系包含$\vb{e},\vb{u},\vb{v},\vb{w}$。假设空间中有一个点$\vb{p}$,在两个坐标系下的坐标分别是$(x_p,y_p,z_p)$和$(u_p,v_p,w_p)$,我们应该如何在两者间转换?首先,从$uvw$系到$xyz$系是很容易的,这基本上就是一些加法
\begin{Equation}
    \vb{p}=\vb{e}+u_p\vb{u}+v_p\vb{v}+w_p\vb{w}
\end{Equation}

我们可以用矩阵的方式来表示
\begin{Equation}[坐标系变换1]
    \begin{pmatrix}
        x_p\\
        y_p\\
        z_p\\
        1
    \end{pmatrix}=
    \begin{pmatrix}
        x_u&x_v&x_w&x_e\\
        y_u&y_v&y_w&y_e\\
        z_u&z_v&z_w&z_e\\
        0&0&0&1\\
    \end{pmatrix}
    \begin{pmatrix}
        u_p\\
        v_p\\
        w_p\\
        1
    \end{pmatrix}
\end{Equation}

我们可以更紧凑的写为
\begin{Equation}[坐标系变换2]
    \vb{p}_{xyz}=\begin{pmatrix}
        \vb{u}&\vb{v}&\vb{w}&\vb{e}\\
        0&0&0&1\\
    \end{pmatrix}
    \vb{p}_{uvw}
\end{Equation}

现在的问题是,通常遇到的坐标变换问题并不是这个方向的。我们往往是期望从$xyz$坐标系变换到$uvw$坐标系,该过程是\cref{eq:坐标系变换2}过程的逆过程,因此变换矩阵要变为对应的逆矩阵
\begin{Equation}[坐标系变换3]
    \vb{p}_{uvw}=\begin{pmatrix}
        \vb{u}&\vb{v}&\vb{w}&\vb{e}\\
        0&0&0&1\\
    \end{pmatrix}^{-1}
    \vb{p}_{xyz}
\end{Equation}

\begin{Figure}[坐标系变换]
    \includegraphics[scale=0.8]{Chapter06E_01.fig.pdf}
\end{Figure}

然而一般矩阵求逆是很麻烦的,这里有一个更简单的办法,回到\cref{eq:坐标系变换1},拆开仿射矩阵
\begin{Equation}[坐标系变换4]
    \begin{pmatrix}
        x_p\\
        y_p\\
        z_p\\
        1
    \end{pmatrix}=
    \begin{pmatrix}
        1&0&0&x_e\\
        0&1&0&y_e\\
        0&0&1&z_e\\
        0&0&0&1\\
    \end{pmatrix}
    \begin{pmatrix}
        x_u&x_v&x_w&0\\
        y_u&y_v&y_w&0\\
        z_u&z_v&z_w&0\\
        0&0&0&1\\
    \end{pmatrix}
    \begin{pmatrix}
        u_p\\
        v_p\\
        w_p\\
        1
    \end{pmatrix}
\end{Equation}

现在我们求逆,我们知道$(\vb{A}\vb{B})^{-1}=\vb{B}^{-1}\vb{A}^{-1}$,因此分别求逆时要交换顺序。随后,平移变换矩阵求逆实际上就是将每一个平移分量变为对应负值,线性变换矩阵是一个正交矩阵(考虑到$\vb{u},\vb{v},\vb{w}$都是单位向量且相互垂直),而正交矩阵的逆矩阵就是其转置矩阵,因此有
\begin{Equation}[坐标系变换5]
    \begin{pmatrix}
        u_p\\
        v_p\\
        w_p\\
        1
    \end{pmatrix}
    =
    \begin{pmatrix}
        x_u&y_u&z_u&0\\
        x_v&y_v&z_v&0\\
        x_w&y_w&z_w&0\\
        0&0&0&1\\
    \end{pmatrix}
    \begin{pmatrix}
        1&0&0&-x_e\\
        0&1&0&-y_e\\
        0&0&1&-z_e\\
        0&0&0&1\\
    \end{pmatrix}
    \begin{pmatrix}
        x_p\\
        y_p\\
        z_p\\
        1
    \end{pmatrix}
\end{Equation}

注意,这里不能再合并回仿射矩阵的形式了,仿射一定是先线性变换再平移变换!

将该结论整理如下
\begin{BoxFormula}[坐标系变换]
    由$xyz$坐标系至$uvw$坐标系的变换矩阵可以表示为
    \begin{Equation}
        \vb{M}=
        \begin{pmatrix}
            x_u&y_u&z_u&0\\
            x_v&y_v&z_v&0\\
            x_w&y_w&z_w&0\\
            0&0&0&1\\
        \end{pmatrix}
        \begin{pmatrix}
            1&0&0&-x_e\\
            0&1&0&-y_e\\
            0&0&1&-z_e\\
            0&0&0&1\\
        \end{pmatrix}
    \end{Equation}
\end{BoxFormula}

% 这一节,我们来考虑一个问题,如何从一个坐标系变换到另一个坐标系?具体来说,我们会考虑一个点$\vb{p}$,它在$xyz$坐标系和$uvw$坐标系下分别被记为$\vb{p}_{xyz}$和$\vb{p}_{uvw}$
% \begin{Equation}
%     \vb{p}_{xyz}=
%     \begin{pmatrix}
%         x_p\\
%         y_p\\
%         z_p\\
%         1\\
%     \end{pmatrix}\qquad
%     \vb{p}_{uvw}=
%     \begin{pmatrix}
%         u_p\\
%         y_p\\
%         z_p\\
%         1    
%     \end{pmatrix}
% \end{Equation}
% 在$xyz$坐标系下,$xyz$坐标系的原点$\vb{o}$和基矢$\vb{x},\vb{y},\vb{z}$为
% \begin{Equation}
%     \vb{o}=
%     \begin{pmatrix}
%         0\\
%         0\\
%         0\\
%         1\\
%     \end{pmatrix}
%     \qquad
%     \vb{x}=
%     \begin{pmatrix}
%         1\\
%         0\\
%         0\\
%         0\\
%     \end{pmatrix}
%     \qquad
%     \vb{y}=
%     \begin{pmatrix}
%         0\\
%         1\\
%         0\\
%         0\\
%     \end{pmatrix}
%     \qquad
%     \vb{z}=
%     \begin{pmatrix}
%         0\\
%         0\\
%         1\\
%         0\\
%     \end{pmatrix}
% \end{Equation}
% 在$xyz$坐标系下,$uvw$坐标系的原点$\vb{e}$和基矢$\vb{u},\vb{v},\vb{w}$为
% \begin{Equation}
%     \vb{e}=
%     \begin{pmatrix}
%         x_e\\
%         y_e\\
%         z_e\\
%         1\\
%     \end{pmatrix}
%     \qquad
%     \vb{u}=
%     \begin{pmatrix}
%         x_u\\
%         y_u\\
%         z_u\\
%         1\\
%     \end{pmatrix}
%     \qquad
%     \vb{v}=
%     \begin{pmatrix}
%         x_v\\
%         y_v\\
%         z_v\\
%         1\\
%     \end{pmatrix}
%     \qquad
%     \vb{w}=
%     \begin{pmatrix}
%         x_w\\
%         y_w\\
%         z_w\\
%         1\\
%     \end{pmatrix}
% \end{Equation}

% 这样可以有
% \begin{Equation}
%     \vb{p}_{xyz}=\vb{o}+x_p\vb{x}+y_p\vb{y}+z_p\vb{z}=\vb{e}+u_p\vb{u}+v_p\vb{v}+w_p\vb{w}
% \end{Equation}
% 先解决一个问题,如何由$\vb{p}_{uvw}$得到$\vb{p}_{xyz}$?这很简单,只要代入$\vb{u},\vb{v},\vb{w},\vb{e}$就可以了
% \begin{Equation}
%     \begin{pmatrix}
%         x_p\\
%         y_p\\
%         z_p\\
%         1
%     \end{pmatrix}=
%     \begin{pmatrix}
%         x_u&x_v&x_w&x_e\\
%         y_u&y_v&y_w&y_e\\
%         z_u&z_v&z_w&z_e\\
%         0&0&0&1\\
%     \end{pmatrix}
%     \begin{pmatrix}
%         u_p\\
%         v_p\\
%         w_p\\
%         1
%     \end{pmatrix}
% \end{Equation}
% 或者更紧凑的写为
% \begin{Equation}
%     \vb{p}_{xyz}=
%     \begin{pmatrix}
%         \vb{u}&\vb{v}&\vb{w}&\vb{e}
%     \end{pmatrix}
%     \vb{p}_{uvw}
% \end{Equation}