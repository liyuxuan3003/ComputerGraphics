\section{Blinn-Phong着色}
现实中许多物体看起来是有光泽的。这是因为,物体表面受到光照时,在散射(Diffuse)机制外还同时存在镜面反射(Specular Reflection)机制。换言之,如果视角在光源产生的入射光对应的反射光方向上,看到的颜色会更亮。Blinn-Phong着色就是在考虑镜面反射的影响。

如\xref{fig:Lambertian着色}所示,$\vb{n}$是表面的法向,$\vb{l},\vb{v}$分别为光源和视线的方向。这里要考察视线方向是否位于光源入射光的反射光上,其实一个等效的做法是,考察$\vb{v},\vb{l}$平分线处的单位矢量$\vb{h}$和表面的法向$\vb{n}$间的夹角$\alpha$有多大?容易证明,我们可以将$\vb{h}$可以表达成以下形式
\begin{Equation}
    \vb{h}=\frac{\vb{v}+\vb{l}}{\norm{\vb{v}+\vb{l}}}
\end{Equation}
因为$\vb{v},\vb{l}$均为单位矢量故$\vb{v}+\vb{l}$必位于两者平分线上,而除以模是为将其也转为单位矢量。

\begin{Figure}[Blinn-Phong着色]
    \includegraphics[scale=0.8]{build/Chapter05C_01.fig.pdf}
\end{Figure}
基于此,Blinn-Phong着色可以表述为
\begin{BoxFormula}[Blinn-Phong着色]
    Blinn-Phong着色
    \begin{Equation}
        c=c_ac_r+c_lc_r\max(0,\vb{n}\cdot\vb{l})+c_lc_p\max(0,\vb{n}\cdot\vb{h})^p
    \end{Equation}
\end{BoxFormula}
其中,$c_p$是镜面反射颜色,$c_p$和$c_r$不一定要是相同的,但是应当确保$c_p+c_r<1$以保证不会令RGB溢出$[0,1]$的范围。$p$称为Phong指数,它代表了物体表面的光泽度,我们容易看出,$p$越大则镜面反射项$c_lc_p\max(0,\vb{n}\cdot\vb{h})^p$衰减越快,可以这样理解,镜面反射会令视线处于反射光两侧一定角度内出现额外的高光,$p$则会决定高光的范围,$p$越大则高光范围越小。

在\xref{tab:着色过程的符号和表达式整理}中,我们整理了着色中出现的符号和表达式
\begin{Tablex}[着色过程的符号和表达式整理]{llZ}
    <颜色或表达式&&说明\\>
    $c_r$&&物体表面的散射颜色\\
    $c_p$&&物体表面的反射颜色\\
    $c_l$&&点状光源的颜色\\
    $c_a$&&环境光源的颜色\\
    $c_r+c_p\leq 1$&&物体的颜色和不能超过$1$\\
    $c_l+c_a\leq 1$&&光源的颜色和不能超过$1$\\ \hline
    $c_ac_r$&Ambient&环境光源的散射\\
    $c_lc_r\max(0,\vb{n}\cdot\vb{l})$&Lambertian&点光源的散射\\
    $c_lc_p\max(0,\vb{n}\cdot\vb{h})^{p}$&Blinn-Phong&点光源的反射\\
\end{Tablex}