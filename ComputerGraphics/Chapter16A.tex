\section{隐式函数的构成}
通常来说,我们可以将一个隐式函数$f_i(\vb{p})$拆分为两部分
\begin{BoxDefinition}[隐式函数的构成]
    隐式函数可以拆分为衰减滤波函数和距离函数的复合
    \begin{Equation}
        f_i(\vb{p})=g_i(d)\circ d_i(\vb{p})
    \end{Equation}
\end{BoxDefinition}
在\xref{def:隐式函数的构成}中
\begin{itemize}
    \item 距离函数(Distance Function)$d_i(\vb{p})$代表空间中一点$\vb{p}$到第$i$个骨架(Skeleton)元素的距离。隐式曲面往往是围绕一些简单几何图形组成的骨架生成的,试想,球面就是由球心所在点生成的。最简单的骨架就是点集,此时,对于其中第$i$个点$(x_i,y_i,z_i)$,距离函数就是$\vb{p}=(x,y,z)$和该点的距离$d_i(\vb{p})=\sqrt{(x-x_i)^2+(y-y_i)^2+(z-z_i)^2}$。更复杂的骨架除了点,还可以包含直线段、圆环、圆盘等元素,此时距离函数$d_i(\vb{p})$会更复杂。
    \item 衰减滤波函数(Fall-Off Fliter Function)$g_i(d)$可以将距离映射到$[0,1]$的范围内,特别的,其要保证$\vb{p}$在骨架上时$g_i(0)=1$,且随着$d$的增大$g_i(d)$逐渐减小,即所谓“衰减”。
\end{itemize}

那么现在的问题是,有哪些可用的衰减滤波函数呢?常用的有以下这些。
\begin{BoxFormula}[Blinn衰减滤波函数]
    Blinn衰减滤波函数为
    \begin{Equation}
        g(d)=\exp(-rd^2)
    \end{Equation}
\end{BoxFormula}

\begin{BoxFormula}[Metaballs衰减滤波函数]
    Metaballs衰减滤波函数为
    \begin{Equation}
        g(d)=\begin{cases}
            1-3\qty(\dfrac{d}{r})^2,&0\leq d\leq r/3\\[5mm]
            \dfrac{3}{2}\qty(1-\dfrac{d}{r})^2,&r/3\leq d\leq r
        \end{cases}
    \end{Equation}
\end{BoxFormula}

\begin{BoxFormula}[Soft Objects衰减滤波函数]
    Soft Objects衰减滤波函数为
    \begin{Equation}
        g(d)=\qty(1-\frac{4d^6}{9r^6}+\frac{17d^4}{9r^4}-\frac{22d^2}{9r^2})
    \end{Equation}
\end{BoxFormula}

\begin{BoxFormula}[Wyvill衰减滤波函数]
    Wyvill衰减滤波函数为
    \begin{Equation}
        g(d)=\qty(1-\frac{d^2}{r^2})^3
    \end{Equation}
\end{BoxFormula}

在\xref{fig:衰减滤波函数}中,我们列出了上述四种衰减滤波函数的图像,取$r=1$,这里$r$可用认为是一个与半径有关的参数。Metaballs,Soft Objects,Wyvill都仅适用于$d\in[0,r]$的范围,故对于它们而言,半径$r$代表了其最大作用范围。Blinn使用了指数函数,适用于$d\in[0,\infty)$的范围。
\begin{Figure}[衰减滤波函数]
    \includegraphics[scale=0.8]{build/Chapter16A_01a.fig.pdf}
\end{Figure}

这里的问题是,我们引入衰减滤波函数$g_i(d)$的意义是什么?相较$f_i(\vb{p})=g_i(d_i(\vb{p}))$,为何不能直接将距离函数作为隐函数$f_i(\vb{p})=d_i(\vb{p})$?这其实是为了方便将多个骨架单元的隐函数混合在一起。混合(Blending)是指,对于两个骨架单元的隐函数$f_A(\vb{p}),f_B(\vb{p})$,如何构建一个$f(\vb{p})$能将$f_A(\vb{p}),f_B(\vb{p})$所代表的隐式曲面以一定方式连接在一起,作为整个骨架的隐式曲面。\goodbreak

这里先给出最简单的混合方法,即简单的将两个函数相加\nopagebreak
\begin{BoxFormula}[加法混合]
    加法混合是指
    \begin{Equation}
        f(\vb{p})=f_A(\vb{p})+f_B(\vb{P})
    \end{Equation}
\end{BoxFormula}

我们知道,对于一个隐函数$f(\vb{p})$,可以通过以下方式定义一个相关的隐式曲面。其中$\te{iso}$的含义是等值曲面(Isosurface),由于衰减滤波函数的限制,这里$\te{iso}$的取值范围为$\te{iso}\in(0,1)$。
\begin{BoxDefinition}[隐式曲面]
    隐式曲面可由下式定义
    \begin{Equation}
        f(\vb{p})=\te{iso}
    \end{Equation}
\end{BoxDefinition}

我们在\xref{tab:两个逐渐靠近的点构成的曲面}中展示了两个逐渐靠近的点产生的曲面,取$r=1$,采用Soft Objects衰减滤波函数,令$\te{iso}=0.2$。观察到随着距离靠近,原本独立的两个球面逐渐相互吸引并融合在一起。
\begin{Tablex}[两个逐渐靠近的点构成的曲面]{|Y|Y|Y|}
    $(\pm 1.0,0,0)$&$(\pm 0.8,0,0)$&$(\pm 0.6,0,0)$\\ \hlinelig
    \xcell<c>[1ex][-3ex]{\includegraphics[scale=0.68]{build/Chapter16A_02a.fig.pdf}}&
    \xcell<c>[1ex][-3ex]{\includegraphics[scale=0.68]{build/Chapter16A_02b.fig.pdf}}&
    \xcell<c>[1ex][-3ex]{\includegraphics[scale=0.68]{build/Chapter16A_02c.fig.pdf}}\\
    \hlinemid
    $(\pm 0.4,0,0)$&$(\pm 0.2,0,0)$&$(\pm 0.0,0,0)$\\ \hlinelig
    \xcell<c>[1ex][-3ex]{\includegraphics[scale=0.68]{build/Chapter16A_02d.fig.pdf}}&
    \xcell<c>[1ex][-3ex]{\includegraphics[scale=0.68]{build/Chapter16A_02e.fig.pdf}}&
    \xcell<c>[1ex][-3ex]{\includegraphics[scale=0.68]{build/Chapter16A_02f.fig.pdf}}\\
\end{Tablex}