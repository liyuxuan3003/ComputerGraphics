\section{视角变换的流程}

视角变换听上去是一件很复杂的事情:是啊,怎么可能用一个奇妙的矩阵乘一下,三维空间中的点就自己跑到我的屏幕上了!但事实上,视角变换就是矩阵乘法这么简单。当然,我们肯定无法直接理解从三维空间到屏幕的变换,我们会将这个过程拆分为若干个变换的组合。

\xref{fig:视角变换流程}最为清晰的展示了视角变换的流程,整个过程分为四步变换
\begin{enumerate}
    \item 相机变换(Camera Transformation),用$\vb{M}_{cam}$表示。
    \item 透视投影变换(Perspective Projection Transformation),用$\vb{M}_{pers}$表示。
    \item 平行投影变换(Orthographic Projection Transformation),用$\vb{M}_{orth}$表示。
    \item 窗口变换(Viewport Transformation),用$\vb{M}_{vp}$表示。
\end{enumerate}
视角变换的过程可以概述如下:最初,我们位于世界坐标系(World Space)中,有一个相机位于$\vb{e}$处并具有$\vb{u},\vb{v},\vb{w}$的基矢,沿$\vb{w}$负方向进行观察。我们第一步要做的相机变换就是将坐标系沿$\vb{u},\vb{v},\vb{w}$对正(实际上就是\xref{sec:坐标系变换}中$xyz$系至$uvw$系的变换),这样就来到了相机坐标系(Camera Space)中。我们知道,相机观察时有透视投影和平行投影两种方式,对于透视投影,我们的处理方式是将其化归为平行投影,也就是要将\xref{fig:视角变换流程}的(B)中的梯形变为(C)中的矩形(严格来说应该是四棱台到长方体),对于平行投影,这一步自然可以跳过。接下来的问题是,平行投影自身怎么实现?我们会先将其通过简单的缩放和平移变换到一个$R=[-1,1]^3$的规范观察体(Canonical View Volume),随后,我们实现从三维到二维屏幕区域$R=[-0.5,n_x-0.5]\times[-0.5,n_y-0.5]$的窗口变换,不过我们不会丢弃$z$轴数据,而是将其保留下来作为深度信息,在光栅化时帮助我们判断物体的前后遮挡关系(应显示$z$最小的)。

\begin{Figure}[视角变换流程]
    \includegraphics[scale=0.8]{build/Chapter07A_06.fig.pdf}
\end{Figure}