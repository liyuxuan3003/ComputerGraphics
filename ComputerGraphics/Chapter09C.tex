\section{阴影映射}

运用\cref{sec:着色方程}的方法,我们确实可以令物体在不同光照角度呈现出不同的明暗层次,然而,我们却无法实现物体本身产生的阴影。这是因为,计算某点的颜色时,我们并没有考虑该点和光源间是否有其他物体的阻挡。本节将介绍的阴影映射(Shadow Mapping)就将解决该问题。
\begin{Figure}[阴影映射]
    \includegraphics[scale=0.8]{Chapter09C_01.fig.pdf}
\end{Figure}

阴影映射的思想很简单:能被光源照射到的地方一定能被从光源方向看到!因此,我们只需要从光源方向渲染一张图像,但这一次,我们只关心$z$-buffer,它记录了从光源出发各个方向的最近距离$d_{map}$。而着色时,需要先判断该点处与光源的距离$d$是否等于该方向的$d_{map}$
\begin{Equation}
    d-d_{map}<\epsilon
\end{Equation}
其中$\epsilon$是为避免浮点误差添加的数(比如$\epsilon=0.001$),该容差称为阴影偏置(Shadow Bias)。