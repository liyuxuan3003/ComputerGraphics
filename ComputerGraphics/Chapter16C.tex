\section{隐式曲面的渲染方法}

隐式曲面的渲染需要将曲面转换为一系列三角面,这可以用以下方法实现:首先,将整个空间均匀划分为一系列称为体素(Voxel)的小方块,随后,通过一定的算法,找到所有被曲面经过的体素方块,最后,在这些体素方块内取一个或多个三角面来近似表示该体素内的曲面。

除此之外,隐式曲面也可以通过光线追踪直接渲染,只要能恰当的求解相交方程。

\subsection{确定与曲面相交的体素}
首先,最基本的一个问题是,对于一个隐式曲面,如何判断其是否经过一个立方体?
\begin{Equation}
    f(\vb{p})=\te{iso}
\end{Equation}

如\cref{fig:确定与曲面相交的体素}所示,我们可以依次令$\vb{p}=V_0,V_1,\cdots,V_7$,对于每个顶点计算出$f(\vb{p})-\te{iso}$的值,并考察该值的正负,若为$-$即顶点在曲面外,若为$+$则顶点在曲面内,因此,如果有一条边的两个顶点异号,则说明曲面一定会经过该边!只要存在异号顶点,就说明曲面经过该立方体。
\begin{Figure}[确定与曲面相交的体素]
    \includegraphics[scale=0.8]{Chapter16C_01.fig.pdf}
\end{Figure}
因此,找到所有与曲面相交的体素的算法就可以概括如下:首先,遍历所有体素,直到找到任意一个与曲面相交的体素(存在异号顶点)。随后,以该体素为搜索起点,找出该体素上与曲面相交的边,并将该体素上所有包含这些边的面临接的体素加入待搜索列表,但不要重复加入已被搜索过的体素。只要迭代这一搜索过程,所有与曲面相交的体素就都可以都被找到了!

\subsection{确定体素内的近似三角面}
接下来,我们来研究如何确定三角面。如\cref{fig:确定体素内的近似三角面}所示,我们假设这八个顶点分别具有图中所示的$f(\vb{p})-\te{iso}$的值,对于那些具有异号顶点的边,可以根据其顶点值的比例,插值出曲面与该边交点的近似位置。随后,将所有交点连在一起,我们就得到了该体素内曲面的近似三角面。
\begin{Figure}[确定体素内的近似三角面]
    \includegraphics[scale=0.8]{Chapter16C_02.fig.pdf}
\end{Figure}
然而,这一方法在体素内有一个以上的异号顶点时会存在一些疑难问题,如\cref{fig:确定体素内的近似三角面的疑难问题}所示
\begin{itemize}
    \item 若有两个相邻的异号顶点,此时在边上会产生四个交点。然而,这四个交点未必是共面的,这就产生一个问题,我们应该按哪一对角线将其分割为两个三角面?结果是不同的。
    \item 若有两个对角的异号顶点,此时两者会构成两个独立的三角面。然而,这两个三角面之间的四边形是否应当也填充三角面?缺失该面中央的采样使这一问题没有确定的解答。
\end{itemize}
\begin{Figure}[确定体素内的近似三角面的疑难问题]
    \figuresub[两个相邻的顶点]{\includegraphics[scale=0.8]{Chapter16C_03.fig.pdf}}    \hspace{0.75cm}
    \figuresub[两个对角的顶点]{\includegraphics[scale=0.8]{Chapter16C_04.fig.pdf}}    
\end{Figure}
通过一些方法可以避免部分问题,例如,可以将体素的立方体拆分为六个四面体,并考虑每个四面体与曲面的交点。由于四面体任意两个顶点都有边相连,这就避免了\cref{fig:两个对角的顶点}的问题。

% \begin{itemize}
%     \item 将整个空间划分为一系列小方块,称为体素(Voxel)。
%     \item 计算$\vb{f}(p)$在各体素顶点的值,若两顶点的符号相反,则曲面经过其连接的边。
%     \item 计算$\vb{f}(p)$在各边上的相交位置,可以通过顶点处的值插值近似确定。
%     \item 连接一个体素内边上的交点,构成一个或多个三角面。
% \end{itemize}

% 除此之外,若应用光线追踪,如果能恰当计算光线和隐式曲面的交点,则可以直接渲染。