\section{相机变换}
正如\xref{sec:视角变换的流程}中所说,相机变换就是\xref{sec:坐标系变换}中从$xyz$系到$uvw$系的一个变换,我们可以照抄\xref{fml:坐标系变换}的结论。这里有一个疑问,为什么从世界的$xyz$系变换至相机的$uvw$系后\xref{fig:视角变换流程}中的相机空间的坐标轴仍然用$x,y,z$表示?这可以理解为在完成变换后我们重新命名了坐标轴,或者说,我们总是用$x,y,z$表示一个空间自身的坐标轴,否则每次变换就要引入三个新字母。

\begin{BoxFormula}[相机变换]
    相机变换的变换矩阵$\vb{M}_{cam}$是
    \begin{Equation}
        \vb{M}_{cam}=
        \begin{pmatrix}
            x_u&y_u&z_u&0\\
            x_v&y_v&z_v&0\\
            x_w&y_w&z_w&0\\
            0&0&0&1\\
        \end{pmatrix}
        \begin{pmatrix}
            1&0&0&-x_e\\
            0&1&0&-y_e\\
            0&0&1&-z_e\\
            0&0&0&1\\
        \end{pmatrix}
    \end{Equation}
\end{BoxFormula}

这里还有一个问题,相机的$\vb{u},\vb{v},\vb{w}$如何确定?如\xref{fig:相机基矢的确定}所示,通常关于一个相机我们了解的是它的观察方向$\vb{g}$和一个表示“上方”的$\vb{t}$,这里最容易求出的是$\vb{w}$,它是$\vb{g}$反方向的单位向量
\begin{Equation}
    \vb{w}=-\frac{\vb{g}}{\norm{\vb{g}}}
\end{Equation}
我们知道,在另外两个基矢中,$\vb{u}$代表画面的右,$\vb{v}$代表画面的上,因此或许会理所当然的写出$\vb{v}=\vb{t}/\norm{\vb{t}}$,然而,这里的$\vb{t}$未必恰好在$\vb{v}$的方向上,如\xref{fig:相机基矢的确定},可以看到$\vb{t}$只是指示了一个大致的向上的朝向。因此,更明智的做法是先通过$\vb{t}$和$\vb{w}$构成的平面求出垂直的$\vb{u}$,即
\begin{Equation}
    \vb{u}=\frac{\vb{t}\times\vb{w}}{\norm{\vb{t}\times\vb{w}}}
\end{Equation}

我们现在就可以很容易得到$\vb{v}$了
\begin{Equation}
    \vb{v}=\vb{w}\times\vb{u}
\end{Equation}

\begin{Figure}[相机基矢的确定]
    \includegraphics[scale=0.8]{build/Chapter07B_01.fig.pdf}
\end{Figure}
将该结论整理如下
\begin{BoxFormula}[相机基矢的确定]
    相机的基矢$\vb{u},\vb{v},\vb{w}$可以通过下式确定
    \begin{Equation}
        \vb{w}=-\frac{\vb{g}}{\norm{\vb{g}}}\qquad
        \vb{u}=\frac{\vb{t}\times\vb{w}}{\norm{\vb{t}\times\vb{w}}}\qquad
        \vb{v}=\vb{w}\times\vb{u}
    \end{Equation}
\end{BoxFormula}