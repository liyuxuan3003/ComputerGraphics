\section{坐标系变换}
在这一节,我们讨论有关坐标系变换的问题,如\xref{fig:坐标系变换}所示。我们知道,在三维空间下的坐标系可以由其原点和三个基矢表征,$xyz$坐标系包含$\vb{o},\vb{x},\vb{y},\vb{z}$,$uvw$坐标系包含$\vb{e},\vb{u},\vb{v},\vb{w}$。假设空间中有一个点$\vb{p}$,在两个坐标系下的坐标分别是$(x_p,y_p,z_p)$和$(u_p,v_p,w_p)$,我们应该如何在两者间转换?首先,从$uvw$系到$xyz$系是很容易的,这基本上就是一些加法
\begin{Equation}
    \vb{p}=\vb{e}+u_p\vb{u}+v_p\vb{v}+w_p\vb{w}
\end{Equation}

我们可以用矩阵的方式来表示\setpeq{坐标系变换}
\begin{Equation}&[1]
    \begin{pmatrix}
        x_p\\
        y_p\\
        z_p\\
        1
    \end{pmatrix}=
    \begin{pmatrix}
        x_u&x_v&x_w&x_e\\
        y_u&y_v&y_w&y_e\\
        z_u&z_v&z_w&z_e\\
        0&0&0&1\\
    \end{pmatrix}
    \begin{pmatrix}
        u_p\\
        v_p\\
        w_p\\
        1
    \end{pmatrix}
\end{Equation}

我们可以更紧凑的写为
\begin{Equation}&[2]
    \vb{p}_{xyz}=\begin{pmatrix}
        \vb{u}&\vb{v}&\vb{w}&\vb{e}\\
        0&0&0&1\\
    \end{pmatrix}
    \vb{p}_{uvw}
\end{Equation}

现在的问题是,通常遇到的坐标变换问题并不是这个方向的。我们往往是期望从$xyz$坐标系变换到$uvw$坐标系,该过程是\xrefpeq{2}过程的逆过程,因此变换矩阵要变为对应的逆矩阵
\begin{Equation}&[3]
    \vb{p}_{uvw}=\begin{pmatrix}
        \vb{u}&\vb{v}&\vb{w}&\vb{e}\\
        0&0&0&1\\
    \end{pmatrix}^{-1}
    \vb{p}_{xyz}
\end{Equation}
然而一般矩阵求逆是很麻烦的,这里有一个更简单的办法,回到\xrefpeq{1},拆开仿射矩阵
\begin{Equation}&[4]
    \begin{pmatrix}
        x_p\\
        y_p\\
        z_p\\
        1
    \end{pmatrix}=
    \begin{pmatrix}
        1&0&0&x_e\\
        0&1&0&y_e\\
        0&0&1&z_e\\
        0&0&0&1\\
    \end{pmatrix}
    \begin{pmatrix}
        x_u&x_v&x_w&0\\
        y_u&y_v&y_w&0\\
        z_u&z_v&z_w&0\\
        0&0&0&1\\
    \end{pmatrix}
    \begin{pmatrix}
        u_p\\
        v_p\\
        w_p\\
        1
    \end{pmatrix}
\end{Equation}
现在我们求逆,我们知道$(\vb{A}\vb{B})^{-1}=\vb{B}^{-1}\vb{A}^{-1}$,因此分别求逆时要交换顺序。随后,平移变换矩阵求逆实际上就是将每一个平移分量变为对应负值,线性变换矩阵是一个正交矩阵(考虑到$\vb{u},\vb{v},\vb{w}$都是单位向量且相互垂直),而正交矩阵的逆矩阵就是其转置矩阵,因此有
\begin{Equation}&[5]
    \begin{pmatrix}
        u_p\\
        v_p\\
        w_p\\
        1
    \end{pmatrix}
    =
    \begin{pmatrix}
        x_u&y_u&z_u&0\\
        x_v&y_v&z_v&0\\
        x_w&y_w&z_w&0\\
        0&0&0&1\\
    \end{pmatrix}
    \begin{pmatrix}
        1&0&0&-x_e\\
        0&1&0&-y_e\\
        0&0&1&-z_e\\
        0&0&0&1\\
    \end{pmatrix}
    \begin{pmatrix}
        x_p\\
        y_p\\
        z_p\\
        1
    \end{pmatrix}
\end{Equation}
注意,这里不能再合并回仿射矩阵的形式了,仿射一定是先线性变换再平移变换!
\begin{Figure}[坐标系变换]
    \includegraphics[scale=0.8]{build/Chapter06E_01.fig.pdf}
\end{Figure}
将该结论整理如下
\begin{BoxFormula}[坐标系变换]
    由$xyz$坐标系至$uvw$坐标系的变换矩阵可以表示为
    \begin{Equation}
        \vb{M}=
        \begin{pmatrix}
            x_u&y_u&z_u&0\\
            x_v&y_v&z_v&0\\
            x_w&y_w&z_w&0\\
            0&0&0&1\\
        \end{pmatrix}
        \begin{pmatrix}
            1&0&0&-x_e\\
            0&1&0&-y_e\\
            0&0&1&-z_e\\
            0&0&0&1\\
        \end{pmatrix}
    \end{Equation}
\end{BoxFormula}

% 这一节,我们来考虑一个问题,如何从一个坐标系变换到另一个坐标系?具体来说,我们会考虑一个点$\vb{p}$,它在$xyz$坐标系和$uvw$坐标系下分别被记为$\vb{p}_{xyz}$和$\vb{p}_{uvw}$
% \begin{Equation}
%     \vb{p}_{xyz}=
%     \begin{pmatrix}
%         x_p\\
%         y_p\\
%         z_p\\
%         1\\
%     \end{pmatrix}\qquad
%     \vb{p}_{uvw}=
%     \begin{pmatrix}
%         u_p\\
%         y_p\\
%         z_p\\
%         1    
%     \end{pmatrix}
% \end{Equation}
% 在$xyz$坐标系下,$xyz$坐标系的原点$\vb{o}$和基矢$\vb{x},\vb{y},\vb{z}$为
% \begin{Equation}
%     \vb{o}=
%     \begin{pmatrix}
%         0\\
%         0\\
%         0\\
%         1\\
%     \end{pmatrix}
%     \qquad
%     \vb{x}=
%     \begin{pmatrix}
%         1\\
%         0\\
%         0\\
%         0\\
%     \end{pmatrix}
%     \qquad
%     \vb{y}=
%     \begin{pmatrix}
%         0\\
%         1\\
%         0\\
%         0\\
%     \end{pmatrix}
%     \qquad
%     \vb{z}=
%     \begin{pmatrix}
%         0\\
%         0\\
%         1\\
%         0\\
%     \end{pmatrix}
% \end{Equation}
% 在$xyz$坐标系下,$uvw$坐标系的原点$\vb{e}$和基矢$\vb{u},\vb{v},\vb{w}$为
% \begin{Equation}
%     \vb{e}=
%     \begin{pmatrix}
%         x_e\\
%         y_e\\
%         z_e\\
%         1\\
%     \end{pmatrix}
%     \qquad
%     \vb{u}=
%     \begin{pmatrix}
%         x_u\\
%         y_u\\
%         z_u\\
%         1\\
%     \end{pmatrix}
%     \qquad
%     \vb{v}=
%     \begin{pmatrix}
%         x_v\\
%         y_v\\
%         z_v\\
%         1\\
%     \end{pmatrix}
%     \qquad
%     \vb{w}=
%     \begin{pmatrix}
%         x_w\\
%         y_w\\
%         z_w\\
%         1\\
%     \end{pmatrix}
% \end{Equation}

% 这样可以有
% \begin{Equation}
%     \vb{p}_{xyz}=\vb{o}+x_p\vb{x}+y_p\vb{y}+z_p\vb{z}=\vb{e}+u_p\vb{u}+v_p\vb{v}+w_p\vb{w}
% \end{Equation}
% 先解决一个问题,如何由$\vb{p}_{uvw}$得到$\vb{p}_{xyz}$?这很简单,只要代入$\vb{u},\vb{v},\vb{w},\vb{e}$就可以了
% \begin{Equation}
%     \begin{pmatrix}
%         x_p\\
%         y_p\\
%         z_p\\
%         1
%     \end{pmatrix}=
%     \begin{pmatrix}
%         x_u&x_v&x_w&x_e\\
%         y_u&y_v&y_w&y_e\\
%         z_u&z_v&z_w&z_e\\
%         0&0&0&1\\
%     \end{pmatrix}
%     \begin{pmatrix}
%         u_p\\
%         v_p\\
%         w_p\\
%         1
%     \end{pmatrix}
% \end{Equation}
% 或者更紧凑的写为
% \begin{Equation}
%     \vb{p}_{xyz}=
%     \begin{pmatrix}
%         \vb{u}&\vb{v}&\vb{w}&\vb{e}
%     \end{pmatrix}
%     \vb{p}_{uvw}
% \end{Equation}