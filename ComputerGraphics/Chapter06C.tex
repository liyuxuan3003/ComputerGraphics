\section{三维中的矩阵变换}
三维中的矩阵变换和二维中的情形类似,我们将相关矩阵列举如下。
\begin{BoxDefinition}[三维缩放变换]
    三维缩放变换定义为
    \begin{Equation}
        \te{scale}(s_x,s_y,s_z)=
        \begin{pmatrix}
            s_x&0&0\\
            0&s_y&0\\
            0&0&s_z\\
        \end{pmatrix}
    \end{Equation}
\end{BoxDefinition}

三维中的剪切有沿$x,y,z$三个方向的,每一种剪切都涉及另外两个方向的剪切系数。
\begin{BoxDefinition}[三维剪切变换]
    三维中,沿$x$方向的剪切变换定义为
    \begin{Equation}
        \te{shear}_x(d_y,d_z)=
        \begin{pmatrix}
            1&d_y&d_z\\
            0&1&0\\
            0&0&1\\
        \end{pmatrix}
    \end{Equation}
    三维中,沿$y$方向的剪切变换定义为
    \begin{Equation}
        \te{shear}_y(d_z,d_x)=
        \begin{pmatrix}
            1&0&0\\
            d_x&1&d_z\\
            0&0&1\\
        \end{pmatrix}
    \end{Equation}
    三维中,沿$z$方向的剪切变换定义为
    \begin{Equation}
        \te{shear}_z(d_x,d_y)=
        \begin{pmatrix}
            1&0&0\\
            0&1&0\\
            d_x&d_y&1\\
        \end{pmatrix}
    \end{Equation}
\end{BoxDefinition}

三维中的旋转有一些不一样。我们知道,旋转是围绕一根轴在一个面内进行的,二维的旋转只能在二维平面上进行,故只有一种旋转矩阵,但在三维下旋转则可以绕$x,y,z$轴旋转。
\begin{BoxDefinition}[三维旋转变换]*
    三维中,绕$x$轴的旋转变换定义为
    \begin{Equation}
        \te{rotate}_x(\phi)=
        \begin{pmatrix}
            1&0&0\\
            0&\cos\phi&-\sin\phi\\
            0&\sin\phi&\cos\phi\\
        \end{pmatrix}
    \end{Equation}
    三维中,绕$y$轴的旋转变换定义为
    \begin{Equation}
        \te{rotate}_y(\phi)=
        \begin{pmatrix}
            \cos\phi&0&\sin\phi\\
            0&1&0\\
            -\sin\phi&0&\cos\phi\\
        \end{pmatrix}
    \end{Equation}
    三维中,绕$z$轴的旋转变换定义为
    \begin{Equation}
        \te{rotate}_z(\phi)=
        \begin{pmatrix}
            \cos\phi&-\sin\phi&0\\
            \sin\phi&\cos\phi&0\\
            0&0&1\\
        \end{pmatrix}
    \end{Equation}
\end{BoxDefinition}

三维中的反射有翻转$x,y,z$轴三种,但下标记录的是以什么面进行对称。

\begin{BoxDefinition}[三维旋转变换]*
    三维中,以$y,z$平面为对称面(翻转$x$轴)的反射变换定义为
    \begin{Equation}
        \te{reflect}_{yz}=
        \begin{pmatrix}
            -1&0&0\\
            0&1&0\\
            0&0&1\\
        \end{pmatrix}
    \end{Equation}
    三维中,以$z,x$平面为对称面(翻转$y$轴)的反射变换定义为
    \begin{Equation}
        \te{reflect}_{zx}=
        \begin{pmatrix}
            1&0&0\\
            0&-1&0\\
            0&0&1\\
        \end{pmatrix}
    \end{Equation}
    三维中,以$x,y$平面为对称面(翻转$z$轴)的反射变换定义为
    \begin{Equation}
        \te{reflect}_{xy}=
        \begin{pmatrix}
            1&0&0\\
            0&1&0\\
            0&0&-1\\
        \end{pmatrix}
    \end{Equation}
\end{BoxDefinition}

\subsection{法向量变换}

在三维空间中,我们经常会用法向量表示一个面的方向,然而,法向量进行变换后往往不再是法向量。\xref{fig:法向量在三维变换中的问题}展示了这一问题,矩形经过变换矩阵$\vb{M}$的剪切变换后,法向量$\vb{n}$顺着剪切变为了斜向上的$\vb{M}\vb{n}$,但其实$\vb{M}\vb{n}$并不垂直于平面,此时,正确的法向量应是图中标出的$\vb{N}\vb{n}$。

这就告诉我们,当图形通过矩阵$\vb{M}$变换时,法向量若也通过矩阵$\vb{M}$变换将得到不正确的结果,法向量需要另外一个不同的矩阵$\vb{N}$矩形进行变换。故现在的问题是怎么由$\vb{M}$求出$\vb{N}$?



\begin{Figure}[法向量在三维变换中的问题]
    \begin{FigureSub}[变换前;变换前--法向量在三维变换中的问题]
        \includegraphics[scale=0.8]{build/Chapter06C_01.fig.pdf}
    \end{FigureSub}
    \quad
    \begin{FigureSub}[变换后;变换后--法向量在三维变换中的问题]
        \includegraphics[scale=0.8]{build/Chapter06C_02.fig.pdf}
    \end{FigureSub}
\end{Figure}
首先,可以确定的是变换前法向量$\vb{n}$一定垂直于切向量$\vb{t}$\setpeq{法向量变换}
\begin{Equation}&[1]
    \vb{n}^T\vb{t}=0
\end{Equation}

我们将变换后的向量用$\vb{n}_N=\vb{N}\vb{n},~\vb{n}_M=\vb{M}\vb{n},~\vb{t}_M=\vb{M}\vb{t}$表示,我们要找到$\vb{N}$使下式成立
\begin{Equation}&[2]
    \vb{n}_N^T\vb{t}_M=0
\end{Equation}
我们可以在\xrefpeq{1}做一些变形,利用$\vb{M}^{-1}\vb{M}=\vb{I}$
\begin{Equation}
    \vb{n}^T\vb{t}=\vb{n}^T\vb{I}\vb{t}=\vb{n}^T\vb{M}^{-1}\vb{M}\vb{t}=0
\end{Equation}
应用矩阵乘法的分配律
\begin{Equation}
    (\vb{n}^T\vb{M}^{-1})(\vb{M}\vb{t})=0
\end{Equation}
注意到$\vb{M}\vb{t}$就是$\vb{t}_M$
\begin{Equation}
    (\vb{n}^T\vb{M}^{-1})\vb{t}_M=0
\end{Equation}
比照\xrefpeq{2},即得
\begin{Equation}
    \vb{n}_N^T=\vb{n}^T\vb{M}^{-1}
\end{Equation}
两边转置
\begin{Equation}
    \vb{n}_N=(\vb{M}^{-1})^T\vb{n}
\end{Equation}

由此可见应有$\vb{N}=(\vb{M}^{-1})^T$,即适用法向量变换的$\vb{N}$是$\vb{M}$的逆矩阵的转置。
\begin{BoxFormula}[法向量变换]
    法向量的变换矩阵$\vb{N}$为
    \begin{Equation}
        \vb{N}=(\vb{M}^{-1})^T
    \end{Equation}
\end{BoxFormula}

我们可以证明,这里$\vb{N}$可以写为
\begin{Equation}
    \vb{N}=
    \begin{pmatrix}
        m_{11}^c&m_{12}^c&m_{13}^c\\
        m_{21}^c&m_{22}^c&m_{23}^c\\
        m_{31}^c&m_{32}^c&m_{33}^c\\    
    \end{pmatrix}
\end{Equation}
其中$m_{ij}^c$代表$\vb{M}$在$m_{ij}$处的代数余子式,完整展开的结果是
\begin{Equation}
    \vb{N}=
    \begin{pmatrix}
        m_{22}m_{33}-m_{23}m_{32}&
        m_{23}m_{31}-m_{21}m_{33}&
        m_{21}m_{32}-m_{22}m_{31}\\
        m_{13}m_{32}-m_{12}m_{33}&
        m_{11}m_{33}-m_{13}m_{31}&
        m_{12}m_{31}-m_{11}m_{32}\\
        m_{12}m_{23}-m_{13}m_{22}&
        m_{13}m_{21}-m_{11}m_{23}&
        m_{11}m_{22}-m_{12}m_{21}\\
    \end{pmatrix}
\end{Equation}
实际上,严格来说上式还要再除以$\vb{M}$的行列式才是真正的$\vb{N}=(\vb{M}^{-1})^T$,但由于法向量并不关心长度(反正最终都要对$\vb{N}\vb{n}$归一化),计算$\vb{N}=(\vb{M}^{-1})^T$时省略除行列式不影响结果。