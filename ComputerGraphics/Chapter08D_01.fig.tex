\makeatletter\def\input@path{{minimus}{standalone-silicon}}\makeatother

\documentclass{standalone-silicon}

\usepackage{minimus-text}
\usepackage{minimus-math}
\usepackage{minimus-tikz}

\begin{document}
\begin{tikzpicture}[scale=1.1]

\draw[ultra thin] (-3.5,-2) rectangle (3,3.5);

\begin{scope}[3d view,3d view={-15}{60}]

\fill[red,fill opacity=0.1] (2,0,0) -- (-1,1.73,0) -- (0,0,1) -- cycle;

\fill[blue,fill opacity=0.1] (2,0,0) -- (-1,-1.73,0) -- (0,0,1);

\fill[green,fill opacity=0.1] (-1,1.73,0) -- (-1,-1.73,0) -- (0,0,1);

\draw (2,0,0) -- (-1,1.73,0) -- (-1,-1.73,0) -- cycle;
\draw (0,0,1) -- (2,0,0);
\draw (0,0,1) -- (-1,1.73,0);
\draw (0,0,1) -- (-1,-1.73,0);

\draw[thick,-latex] (0.333,0.577,0.333) coordinate(A) -- ++ (0.707,1.226,1.414) node[above] {$\vb{n}_0$};
\draw[thick,-latex] (-0.667,0,0.333) coordinate(B) -- ++ (-1.414,0,1.414) node[left] {$\vb{n}_1$};
\draw[thick,-latex] (0.333,-0.577,0.333) coordinate(C) -- ++ (0.707,-1.226,1.414) node[right] {$\vb{n}_2$};

\draw[thick,-latex] (0,0,1) coordinate (T) -- ++(0,0,2)node[left] {$\vb{n}$};

\path (A) node[ocirc,scale=0.8] {};
\path (B) node[ocirc,scale=0.8] {};
\path (C) node[ocirc,scale=0.8] {};

\end{scope}

\end{tikzpicture}
\end{document}