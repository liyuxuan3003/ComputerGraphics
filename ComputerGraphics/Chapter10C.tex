\section{卷积在图像处理中的应用}

卷积在图像处理中有许多应用,如\cref{fig:卷积在图像处理中的应用}所示,最简单的例子就是图像的模糊和锐化。

\begin{Figure}[卷积在图像处理中的应用]
    \figuresub[模糊;卷积图像处理-模糊]{\includegraphics[scale=1.8]{Chapter10B_04a.fig.pdf}}{0.32}
    \figuresub[原始;卷积图像处理-原始]{\includegraphics[scale=1.8]{Chapter10B_04b.fig.pdf}}{0.32}
    \figuresub[锐化;卷积图像处理-锐化]{\includegraphics[scale=1.8]{Chapter10B_04c.fig.pdf}}{0.32}
\end{Figure}

\cref{fig:卷积图像处理-原始}展示的是被栅格化的$e^\pi$的图像,现在尝试对其进行模糊和锐化。\goodbreak

模糊的本质就是与周围的像素做平均,Gaussian卷积核可以提供最均匀的模糊效果
\begin{Equation}
    f_{blur}=f_{g,\sigma}
\end{Equation}

锐化则来自一个很有趣的想法:如果在原始图像中减去模糊图像,那得到图像就应该更为清晰了!设$I$为原始图像,而$I_{blur},I_{sharp}$分别为经模糊和锐化的图像,则可以将$I_{sharp}$表示为
\begin{Equation}
    I_{sharp}=(1+\alpha)I-\alpha I_{blur}
\end{Equation}
其中$\alpha$是表示锐化程度的系数,代入$I_{blur}=I*f_{blur}=I*f_{g,\sigma}$
\begin{Equation}
    I_{sharp}=(1+\alpha)I-\alpha(I*f_{g,\sigma})
\end{Equation}
合并,其中$d$是二维离散的冲激信号(仅在中央为$1$周围都是$0$)
\begin{Equation}
    I_{sharp}=I*((1+\alpha)d-\alpha f_{g,\sigma})
\end{Equation}
这就得到
\begin{Equation}
    f_{sharp}=(1+\alpha)d-\alpha f_{g,\sigma}
\end{Equation}
其中$\alpha$越大,锐化程度越高,\cref{fig:卷积图像处理-锐化}中取$\alpha=1$。有趣的是,若取$\alpha=0$和$\alpha=-1$时,它实际会给出原始($f_{sharp}=d$)和模糊($f_{sharp}=f_{g,\sigma}$)。故\cref{fig:卷积在图像处理中的应用}也可认为是对应$\alpha=-1,0,1$。

\begin{BoxFormula}[图像的模糊]
    图像的模糊可以用以下卷积核表示
    \begin{Equation}
        f_{blur}=f_{g,\sigma}
    \end{Equation}
\end{BoxFormula}

\begin{BoxFormula}[图像的锐化]
    图像的锐化可以用以下卷积核表示
    \begin{Equation}
        f_{sharp}=(1+\alpha)d-\alpha f_{g,\sigma}
    \end{Equation}
\end{BoxFormula}