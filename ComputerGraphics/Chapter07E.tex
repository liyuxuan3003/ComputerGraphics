\section{窗口变换}
窗口变换本质上就是将规范观察体$[-1,1]^3$中的点沿$z$轴“压扁”再将$x,y$缩放至屏幕区域的$[-0.5,n_x-0.5]\times[-0.5,n_y-0.5]$中,当然,实际上我们不会真正丢弃$z$的数据,尽管屏幕是二维的,但$z$的数据仍然需要保留以供后续判断遮挡关系。因此,窗口变换可以表示为
\begin{BoxFormula}[窗口变换]
    窗口变换的变换矩阵$\vb{M}_{vp}$是
    \begin{Equation}
        \vb{M}_{vp}=\begin{pmatrix}
            n_x/2&0&0&(n_x-1)/2\\
            0&n_y/2&0&(n_y-1)/2\\
            0&0&1&0\\
            0&0&0&1\\
        \end{pmatrix}
    \end{Equation}
\end{BoxFormula}
这里也可以展开来写帮助我们理解
\begin{Gather}[6pt]
    x'=\frac{n_x(x+1)}{2}-\frac{1}{2}\\
    y'=\frac{n_y(y+1)}{2}-\frac{1}{2}
\end{Gather}
这里的含义就是先将$[-1,1]^2$各平移加$1$并除$2$变换至$[0,1]^2$,随后分别乘$n_x,n_y$从而得到$[0,n_x]\times[0,n_y]$的屏幕大小,最后各平移减$1/2$得到$[-0.5,n_x-0.5]\times [-0.5,n_y-0.5]$。

至此,视角变换的矩阵就可以用上述四个步骤的变换矩阵的乘积表示了!
\begin{BoxFormula}[视角变换]
    视角变换的变换矩阵可以表示为
    \begin{Equation}
        \vb{M}=\vb{M}_{vp}\vb{M}_{orth}\vb{M}_{pers}\vb{M}_{cam}
    \end{Equation}
\end{BoxFormula}