\chapter{表面着色}
着色(Shading)是指这样一件事,若某条光线追踪的某个物体,那么这条光线对应的像素应该渲染为什么颜色?最朴素的思路是,物体是什么颜色就将像素渲染为什么颜色,这可行但效果肯定不好,试想,这样渲染一个蓝色的球体将得到一个完全是蓝色的圆,毫无立体感。

到底是什么赋予了物体明暗?光源!朴素的说,正对光的面比较亮,侧对光的面比较暗,除此之外,材料有哑光和高光之分。总而言之,着色要研究的就是光源对于物体自身颜色的影响。

\section{Lambertian着色}

Lambertian着色是最简单的模型,其认为,照射至物体表面的光取决于表面和光源的夹角
\begin{itemize}
    \item 若物体表面正对着光源,则表面是完全照亮的。
    \item 若物体表面平行于光源,则表明是完全黑暗的。
\end{itemize}

\begin{Figure}[Lambertian着色]
    \includegraphics[scale=0.8]{build/Chapter05A_01.fig.pdf}
\end{Figure}

如\xref{fig:Lambertian着色}所示,$\vb{n}$是表面的法向,$\vb{l}$是光源的方向,它们都被设定为单位向量。
\begin{BoxFormula}[Lambertian着色]
    Lambertian着色认为
    \begin{Equation}
        c=c_lc_r\max(0,\vb{n}\cdot\vb{l})
    \end{Equation}
\end{BoxFormula}
其中,$c_l$是光源颜色,$c_r$是物体散射颜色,该公式应当对RGB各计算一次以得到RGB每个分量的结果。在\xref{fig:Lambertian着色}中用$\theta$表示$\vb{n},\vb{l}$的夹角,由于$\vb{n},\vb{l}$都是单位矢量,因此$\vb{n}\cdot\vb{l}=\cos\theta$,这符合前面的预期,当$\theta=0$时给出$1$
即最亮,当$\theta=\pm\pi/2$时给出$0$即最暗。然而我们必须要处理的一种情况是当光源位于表面背后,此时$\vb{n}\cdot\vb{l}<0$为负数,这会违背RGB的数值在区间$[0,1]$的要求,而且直观上“不可能比黑更黑”,因此需要将$\vb{n}\cdot\vb{l}$和$0$之间取$\max$函数。

Lambertian着色的一个特点是它与视角无关。这一点我们可以这样理解:Lambertian着色假定物体表面向各方向均匀散射了从光源接收的能量,所以从各个视角看起来都是一样的。
\section{Ambient着色}
Lambertian着色存在的一个问题是,对于物体背对光源部分的表面,将是完全黑暗的,然而这往往和现实不符。物体背光一侧相较迎光一侧肯定要更暗一些,但不可能是完全黑暗的。

Ambient着色改进了这一点,其叠加了一个环境光源$c_a$项
\begin{BoxFormula}[Ambient着色]
    Ambient着色认为
    \begin{Equation}
        c=c_ac_r+c_lc_r\max(0,\vb{n}\cdot\vb{l})
    \end{Equation}
\end{BoxFormula}
Ambient着色可以认为是模拟了现实中由于各个物体的散射使环境中存在的一个各向同性的基础亮度。此外,要指出的是,我们需要确保环境光源和点光源的总和$c_a+c_l<1$,否则会导致RGB的结果溢出$[0,1]$的范围,另外一种可行的方式是当结果高于$1$将其限定在$1$。
\section{Blinn-Phong着色}
现实中许多物体看起来是有光泽的。这是因为,物体表面受到光照时,在散射(Diffuse)机制外还同时存在镜面反射(Specular Reflection)机制。换言之,如果视角在光源产生的入射光对应的反射光方向上,看到的颜色会更亮。Blinn-Phong着色就是在考虑镜面反射的影响。

如\cref{fig:Lambertian着色}所示,$\vb{n}$是表面的法向,$\vb{l},\vb{v}$分别为光源和视线的方向。这里要考察视线方向是否位于光源入射光的反射光上,其实一个等效的做法是,考察$\vb{v},\vb{l}$平分线处的单位矢量$\vb{h}$和表面的法向$\vb{n}$间的夹角$\alpha$有多大?容易证明,我们可以将$\vb{h}$可以表达成以下形式
\begin{Equation}
    \vb{h}=\frac{\vb{v}+\vb{l}}{\norm{\vb{v}+\vb{l}}}
\end{Equation}
因为$\vb{v},\vb{l}$均为单位矢量故$\vb{v}+\vb{l}$必位于两者平分线上,而除以模是为将其也转为单位矢量。

\begin{Figure}[Blinn-Phong着色]
    \includegraphics[scale=0.8]{Chapter05C_01.fig.pdf}
\end{Figure}
基于此,Blinn-Phong着色可以表述为
\begin{BoxFormula}[Blinn-Phong着色]
    Blinn-Phong着色
    \begin{Equation}
        c=c_ac_r+c_lc_r\max(0,\vb{n}\cdot\vb{l})+c_lc_p\max(0,\vb{n}\cdot\vb{h})^p
    \end{Equation}
\end{BoxFormula}
其中,$c_p$是镜面反射颜色,$c_p$和$c_r$不一定要是相同的,但是应当确保$c_p+c_r<1$以保证不会令RGB溢出$[0,1]$的范围。$p$称为Phong指数,它代表了物体表面的光泽度,我们容易看出,$p$越大则镜面反射项$c_lc_p\max(0,\vb{n}\cdot\vb{h})^p$衰减越快,可以这样理解,镜面反射会令视线处于反射光两侧一定角度内出现额外的高光,$p$则会决定高光的范围,$p$越大则高光范围越小。

在\cref{tab:着色过程的符号和表达式整理}中,我们整理了着色中出现的符号和表达式
\begin{Table}[着色过程的符号和表达式整理]!!
    \begin{tblr}{colspec={llZ}}
    颜色或表达式&&说明\\
    $c_r$&&物体表面的散射颜色\\
    $c_p$&&物体表面的反射颜色\\
    $c_l$&&点状光源的颜色\\
    $c_a$&&环境光源的颜色\\
    $c_r+c_p\leq 1$&&物体的颜色和不能超过$1$\\
    $c_l+c_a\leq 1$&&光源的颜色和不能超过$1$\\ \hline
    $c_ac_r$&Ambient&环境光源的散射\\
    $c_lc_r\max(0,\vb{n}\cdot\vb{l})$&Lambertian&点光源的散射\\
    $c_lc_p\max(0,\vb{n}\cdot\vb{h})^{p}$&Blinn-Phong&点光源的反射\\
    \end{tblr}
\end{Table}
\section{着色频率}
在前面三节中,我们讨论了一个点处的颜色应该如何根据光源和视角的方向计算。那么在实践中,对于一个三角面,我们到底要计算多少点的颜色?这就是着色频率,有三个等级

\begin{enumerate}
    \item 平面着色(Flat Shading):面上各点使用同一颜色。
    \item 顶点着色(Gouraud Shading):面上一点的颜色是顶点颜色的插值。
    \item 像素着色(Phong Shading):面上一点的颜色是顶点法向量在该点插值对应的颜色。
\end{enumerate}


这里需要进一步解释的是顶点着色和像素着色,它们都涉及一个问题,什么是顶点处的法向量?这个提法很怪,因为一个三角面只有一个法向量$\vb{n}$。实际上,顶点处的法向量$\vb{n}_a,\vb{n}_b,\vb{n}_c$定义为该顶点连接的所有三角面的法向量的平均值,这样一来,顶点可以均衡考虑临近三角面的方向从而通过后续插值在三角面之间创建比较自然的颜色过渡,而顶点着色和像素着色的差异就在于它们的插值方式略有不同,为此,我们有必要先解释一下三角面上的插值原理。

我们再次写出三角面的参数方程
\begin{Equation}
    \vb{f}=\vb{a}+\beta(\vb{b}-\vb{a})+\gamma(\vb{c}-\vb{a})
\end{Equation}
我们不妨将$\vb{a}$提出
\begin{Equation}
    \vb{f}=(1-\beta-\gamma)\vb{a}+\beta\vb{b}+\gamma\vb{c}
\end{Equation}
引入$\alpha=1-\beta-\gamma$
\begin{Equation}
    \vb{f}=\alpha\vb{a}+\beta\vb{b}+\gamma\vb{c}
\end{Equation}
三角面上任意一点都可以由$(\alpha,\beta,\gamma)$表示,这称为重心坐标(Barycentric Coordinate)。而如果三个顶点处有某个量$x_a,x_b,x_c$,那么$(\alpha,\beta,\gamma)$处的$x$就可由$x=\alpha x_a+\beta x_b+\gamma x_c$插值。

顶点着色的意思是,根据$\vb{n}_a,\vb{n}_b,\vb{n}_c$先算出顶点处的颜色$c_a,c_b,c_c$,随后插值出颜色
\begin{Equation}
    c=\alpha c_a+\beta c_b+\gamma c_c
\end{Equation}
像素着色的意思是,根据$\vb{n}_a,\vb{n}_b,\vb{n}_c$先插值出待计算点处的法向量$\vb{n}$,随后再通过该点处的法向量$\vb{n}$计算出相应的颜色,假如记颜色和法向量间的函数关系为$f$,则有
\begin{Equation}
    c=f(\alpha \vb{n}_a+\beta \vb{n}_b+\gamma \vb{n}_c)
\end{Equation}
简而言之,顶点着色插值对象为颜色,像素着色插值对象为法向量。像素着色的精度要优于顶点着色,尽管似乎只是颜色计算和插值计算互换了一下顺序,但这样一来,像素着色能赋予三角面上每个点均匀变化的法向量,这对于对法向量极为敏感的Blinn-Phong着色模型很重要。


% \begin{enumerate}
%     \item 频率为三角面(Flat Shading),三角面上各点使用同一颜色。
%     \item 频率为顶点(Gouraud Shading),三角面上一点的颜色是顶点颜色的插值。
%     \item 频率为像素(Phong Shading),三角面上一点的颜色是顶点法向量在该点插值对应的颜色。
% \end{enumerate}