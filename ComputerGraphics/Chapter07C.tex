\section{透视投影变换}
透视投影变换是整个视角变换中最困难的一步。在\xref{fig:视角变换流程}中我们看到,透视投影下的观察区域是一个以原点为锥顶点、以$z=z_n$为顶面、以$z=z_f$为底面,顶面范围为$[x_l,x_r]\times [y_b,y_t]$的四棱台,而我们需要将其变换为$[x_l,x_r]\times[y_b,y_t]\times [z_n,z_f]$的长方体。本质上,如\xref{fig:透视投影变换}所示,假如我们要对$(x,y,z)$点做变换,实际上就是要绘出原点和$(x,y,z)$点间的连线,而该连线与$z=z_n$面交点处的$x'$,$y'$就是我们期望的$x,y$的透视投影变换结果(\xref{fig:透视投影变换}中只绘制了$y,y'$的部分,但$x,x'$完全同理)。通过观察,我们可以预见$x',y'$关于$x,y$的关系式可以写作
\begin{Equation}
    x'=x\frac{z_n}{z}\qquad y'=y\frac{z_n}{z}
\end{Equation}

% 我们要解决的问题可以用\xref{fig:透视投影变换}

\begin{Figure}[透视投影变换]
    \includegraphics[scale=0.8]{build/Chapter07C_01.fig.pdf}
\end{Figure}

这里就遇到了一个麻烦的问题:矩阵变换中无法出现“除以$z$”这样的算式!这个问题可以通过进一步延拓齐次坐标的意义来巧妙解决。在\xref{sec:仿射变换与平移}中,我们为了便于平移变换的表示在向量中引入了一个额外的第四元素,若其代表绝对位置而非相对位置,则该第四维度恒定为$1$
\begin{Equation}[齐次坐标旧定义]
    \begin{pmatrix}
        x\\
        y\\
        z\\
        1\\
    \end{pmatrix}\to
    \begin{pmatrix}
        x\\
        y\\
        z\\
    \end{pmatrix}
\end{Equation}
这里的想法是,现在第四元素可以取任意值$w$,映射后向量各元素都除$w$
\begin{Equation}[齐次坐标新定义]
    \begin{pmatrix}
        x\\
        y\\
        z\\
        w\\
    \end{pmatrix}\to
    \begin{pmatrix}
        x/w\\
        y/w\\
        z/w\\
    \end{pmatrix}
\end{Equation}
值得注意的是,这一延拓与原定义不冲突,若\xrefeq{齐次坐标新定义}中$w=1$则其自然退回至\xrefeq{齐次坐标旧定义}。

在延拓后,现在能做的变换看起来是这样
\begin{Equation}
    \begin{pmatrix}
        \tilde{x}\\
        \tilde{y}\\
        \tilde{z}\\
        \tilde{w}\\
    \end{pmatrix}=
    \begin{pmatrix}
        m_{11}&m_{12}&m_{13}&x_t\\
        m_{21}&m_{22}&m_{23}&y_t\\
        m_{31}&m_{32}&m_{33}&z_t\\
        h_1&h_2&h_3&h_t\\
    \end{pmatrix}
    \begin{pmatrix}
        x\\
        y\\
        z\\
        1\\
    \end{pmatrix}
\end{Equation}

若展开来写
\begin{Gather}
    \title{x}=m_{11}x+m_{12}y+m_{13}z+x_t\\
    \title{y}=m_{21}x+m_{22}y+m_{23}z+y_t\\
    \title{z}=m_{31}x+m_{32}y+m_{33}z+z_t\\
    \title{w}=h_1x+h_2y+h_3z+h_t
\end{Gather}
它对应的实际坐标是
\begin{Gather}[6pt]
    x'=\frac{\tilde{x}}{\tilde{w}}=\frac{m_{11}x+m_{12}y+m_{13}z+x_t}{h_1x+h_2y+h_3z+h_t}\\
    y'=\frac{\tilde{y}}{\tilde{w}}=\frac{m_{21}x+m_{22}y+m_{23}z+y_t}{h_1x+h_2y+h_3z+h_t}\\
    z'=\frac{\tilde{z}}{\tilde{w}}=\frac{m_{31}x+m_{32}y+m_{33}z+z_t}{h_1x+h_2y+h_3z+h_t}
\end{Gather}
我们注意到,这个新变换可以为$x',y',z'$嵌入一个相同的$x,y,z$的线性变换作为分母,这就实现了透视投影中需要的“除以$z$”的运算!我们将这个基于仿射变换拓展的新变换称为投影变换(Projection Transformation)或单应性(Homography)。注意,这里所说的“投影变换”是和“仿射变换”一样的一种抽象的变换,不是指视角变换流程中的一个具体的变换步骤。

\begin{BoxDefinition}[投影变换]
    投影变换矩阵为
    \begin{Equation}
        \vb{M}=
        \begin{pmatrix}
            m_{11}&m_{12}&m_{13}&x_t\\
            m_{21}&m_{22}&m_{23}&y_t\\
            m_{31}&m_{32}&m_{33}&z_t\\
            h_1&h_2&h_3&h_t\\
        \end{pmatrix}
    \end{Equation}
\end{BoxDefinition}

有了这个新工具后,透视投影变换就可以表示为如下形式了
\begin{BoxFormula}[透视投影变换]
    透视投影变换的变换矩阵$M_{pers}$是
    \begin{Equation}
        \vb{M}_{pers}=
        \begin{pmatrix}
            z_n&0&0&0\\
            0&z_n&0&0\\
            0&0&z_n+z_f&-z_nz_f\\
            0&0&1&0\\
        \end{pmatrix}
    \end{Equation}
\end{BoxFormula}

我们可以分析一下为什么这么写
\begin{Gather}[6pt]
    x'=\frac{z_n x}{z}\\
    y'=\frac{z_n y}{z}\\
    z'=\frac{(z_n+z_f)z-z_nz_f}{z}=z_n+z_f-\frac{z_nz_f}{z}
\end{Gather}

我们注意到,$\vb{M}_{pers}$中第一行和第二行的$z_n$实现了对$x,y$做$z_n$倍缩放,$\vb{M}_{pers}$第四行第三列的$1$则借助齐次坐标实现了除以$z$,从而达成$x'=z_nx/z$和$y'=z_ny/y$的目标。但我们要注意到除以$z$的效果会作用在每一个元素上,包括$z$自身,这是不期望的,因为这个过程中我们希望$z$不变即有$z'=z$成立,然而,这是不可能的。$\vb{M}_{pers}$的第三行正是在这一背景下通过缩放和平移构造出了$z'=z_n+z_f-z_nz_f/z$这一略显奇怪的变换式。如\xref{fig:透视变换对z的作用}所示,这个变换的价值在于,它可以确保端点不变和保序!换言之,尽管物体的深度发生了一些变化,但是,更近的物体仍然更近,更远的物体仍然更远,前后遮挡关系没有变化!由于最终$z$的数值只需要用于确定遮挡关系,这一副作用可以接受(保证变换始终能用矩阵表示是更重要的)。

\begin{Figure}[透视变换对$z$的作用;透视变换对z的作用]
    \includegraphics[scale=0.8]{build/Chapter07C_02a.fig.pdf}
\end{Figure}