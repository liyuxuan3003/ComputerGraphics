\section{栅格图像}
在了解栅格设备后,本节我们将讨论有关栅格图像的若干问题。

\subsection{栅格图像的数学建模}
首先我们来讨论连续图像,连续图像可以视为这样一个映射
\begin{Equation}
    I(x,y): R\to V
\end{Equation}
\begin{itemize}
    \item $R$是图像存在的一个矩形区域,因此有$R\subset \R^2$。
    \item $V$是图像像素可能的取值范围,对于灰度图像$V\subset\R^{+}$,对于彩色图像$V\subset(\R^{+})^3$。
\end{itemize}

连续图像的实质是光强在矩形域上的连续分布。栅格图像的像素实际上采样的就是像素所在的小区域上光强的平均值,如\xref{fig:栅格图像的建模}所示,这里有必要明确的是像素的排列顺序,使用$(i,j)$表示第$i$列第$j$行的像素,列自左向右,行自下至上。请注意,用$i$表示列用$j$表示行的设定和通常的习惯是相反的!但这样做有一个好处,即$i,j$可以直接与$x,y$对应,转换时不用颠倒。

\begin{Figure}[栅格图像的建模]
    \includegraphics[scale=0.8]{build/Chapter03B_05.fig.pdf}
\end{Figure}

像素坐标$(i,j)$代表了该像素点的中心位置,若总共有$n_x$列$n_y$行像素,则图像的定义域
\begin{Equation}
    R=[-0.5,n_x-0.5]\times [-0.5,n_y-0.5]
\end{Equation}

\subsection{栅格图像的像素值}
我们知道,在计算机中颜色是通过红、绿、蓝即RGB三种颜色以不同比例混合而成的,如果要考虑到颜色的透明度,那还需要添加一个额外的Alpha通道即所谓RGBA。但现在有一个问题,即对于每一种颜色的亮度,我们应当如何量化呢?下面我们给出两种不同的方法。

高动态范围(HDR, High Dynamic Range),使用浮点数来表示亮度
\begin{Equation}
    I=a\times 2^b
\end{Equation}
低动态范围(LDR, Low Dynamic Range),使用整数来表示亮度
\begin{Equation}
    I=Mc
\end{Equation}
这里$M$表示一个最大亮度(这可能代表显示器能发出或相机能分辨的最亮的光),而$c\in[0,1]$代表一个相对于$M$的亮度,存储时会用以整数表示$c$在$[0,1]$中的分度值,例如假设我们用一个$\SI{8}{bit}$的整数量化,则$0,1,2,\cdots,255$相应表示$c=0,1/255,2/255,\cdots,1$。总结起来说,低动态围就是选定一个亮度范围并用整数表达该范围内一系列均匀间隔的亮度点,高动态范围则是利用浮点数的特性,低亮度下间隔较密,高亮度下间隔较疏但可以动态扩展出很大的亮度范围。实践中,使用整数表示每种颜色的亮度的做法是比较常见的,通常在计算机内部用的就是$\SI{8}{bit}$颜色,故一个RGBA颜色会占用$\SI{32}{bit}$的长度,这刚好是$\SI{32}{bit}$硬件一个字的长度。对于一些专业相机,也会使用$\SI{14}{bit}$或$\SI{16}{bit}$图像以追求照片更为细腻的颜色表现。

容易想象,若设RGB颜色的坐标为$(c_r,c_g,c_b)$,那么所有的颜色都将存在于$V=[0,1]^3$的立方体空间内,在\xref{tab:RGB颜色}中,我们列出了八种最基本的RGB颜色,它们是立方体的八个顶点。

\begin{Tablex}[RGB颜色]{llXrc}
    <颜色中文&颜色英文&备注&坐标&颜色\\>
    黑色&Black&--&$(0,0,0)$&\xcell<c>[1ex][0ex]{\tikz\draw[fill=black] (-0.25,-0.25) rectangle (0.25,0.25);}\\
    红色&Red&--&$(1,0,0)$&\xcell<c>[1ex][0ex]{\tikz\draw[fill=red] (-0.25,-0.25) rectangle (0.25,0.25);}\\
    绿色&Green&--&$(0,1,0)$&\xcell<c>[1ex][0ex]{\tikz\draw[fill=green] (-0.25,-0.25) rectangle (0.25,0.25);}\\
    蓝色&Blue&--&$(0,0,1)$&\xcell<c>[1ex][0ex]{\tikz\draw[fill=blue] (-0.25,-0.25) rectangle (0.25,0.25);}\\
    青色&Cyan&绿和蓝的混合&$(0,1,1)$&\xcell<c>[1ex][0ex]{\tikz\draw[fill=cyan] (-0.25,-0.25) rectangle (0.25,0.25);}\\
    紫色&Magenta&蓝和红的混合~也称为“品红”或“洋红”&$(1,0,1)$&\xcell<c>[1ex][0ex]{\tikz\draw[fill=magenta] (-0.25,-0.25) rectangle (0.25,0.25);}\\
    黄色&Yellow&红和绿的混合&$(1,1,0)$&\xcell<c>[1ex][0ex]{\tikz\draw[fill=yellow] (-0.25,-0.25) rectangle (0.25,0.25);}\\
    白色&White&---&$(1,1,1)$&\xcell<c>[1ex][0ex]{\tikz\draw[fill=white] (-0.25,-0.25) rectangle (0.25,0.25);}\\
\end{Tablex}

当我们讨论透明度的影响时,必须有前景(Foreground)和背景(Background)两张图像。假设前景和后景同一位置的像素的某一颜色分量为$c_f$和$c_b$,且前景该像素透明度为$\alpha$,则
\begin{BoxFormula}[透明度时的叠加公式]
    前后景叠加后的颜色可以表示为
    \begin{Equation}
        c=\alpha c_f+(1-\alpha)c_b
    \end{Equation}
\end{BoxFormula}\goodbreak
我们关注两个边界情况
\begin{itemize}
    \item 当前景$\alpha=0$,有$\alpha=c_b$成立,即意味着前景“完全透明”。
    \item 当前景$\alpha=1$,有$\alpha=c_f$成立,即意味着前景“完全不透明”。
\end{itemize}
这意味着,如果我们希望在RGBA下表示不透明的颜色,我们应当令$\alpha=1$(而不是反之)。

\subsection{显示与伽马校正}
若一个像素的值为$a\in[0,1]$,显示屏最大亮度为$M$,显示屏该像素处的亮度应当为
\begin{Equation}
    I=Ma
\end{Equation}

但实际上,并不是如此,因为显示屏的输入和输出之间存在非线性的关系!这种非线性关系可以近似为幂函数,幂函数的指数用$\gamma$表示,这是一个与显示屏特性有关的量
\begin{Equation}
    I=Ma^{\gamma}
\end{Equation}

\xref{fig:伽马值对显示的影响}展示了$\gamma$的影响,对于当下典型的显示屏,有$\gamma=2.2$。
\begin{Figure}[伽马值对显示的影响]
    \includegraphics[scale=0.8]{build/Chapter03B_07a.fig.pdf}
\end{Figure}

现在的问题是,我们如何测定一个显示屏的$\gamma$值?如\xref{fig:伽马值的测定方法}所示,我们令显示屏显示左侧这样的黑白相间的棋盘图案,假如这种图案足够密集,那从远处看就是灰色的,而且由于这种灰色是通过一半白一半黑直接在空间上混合出来的,我们可以确定其$I/M=0.5$。同时,我们在右侧显示一个值为$a$的纯色,调节$a$的值使左右看起来一致,此时有下式成立
\begin{Equation}
    a^{\gamma}=0.5
\end{Equation}

这样一来
\begin{Equation}
    \gamma=\frac{\ln 0.5}{\ln a}
\end{Equation}

\begin{Figure}[伽马值的测定方法]
    \includegraphics[scale=0.8]{build/Chapter03B_03.fig.pdf}
\end{Figure}

假如我们已知知道$\gamma$,为了消除其影响,可以在输出之前通过$a'=a^{1/\gamma}$进行矫正。

\begin{Figure}[灰度色阶的对比]
    \begin{FigureSub}[校正前]
        \includegraphics[scale=0.8]{build/Chapter03B_01.fig.pdf}
    \end{FigureSub}\\ \vspace{0.25cm}
    \begin{FigureSub}[校正后]
        \includegraphics[scale=0.8]{build/Chapter03B_02.fig.pdf}
    \end{FigureSub}
\end{Figure}

这里有一个问题,为什么显示屏不能设法令$\gamma=1$?在早期CRT显示器的年代,输出和输入确实是存在物理上的非线性关系,而现代的显示器其实是完全线性的,前面提到的$\gamma=2.2$其实是人为引入的!这样做的目的可以通过\xref{fig:灰度色阶的对比}所示的一个小实验理解,\xref{fig:校正前}是未校正的色阶,\xref{fig:校正后}是依照$\gamma=2.2$下$a'=a^{1/\gamma}$校正后的色阶,我们会注意到,校正后的灰度色阶人眼看起来反而没有那么均匀了,但如果我们用仪器测量光强,校正后的光强其实是线性分布的,这是怎么回事?实际上,人眼对于光强的感知是非线性的,简单来说,我们感知的其实是光强的数量级而不是绝对值。而显示屏$\gamma=2.2$恰恰就是为了补偿人眼的非线性,这个数值是通过大量试验得出的。所以如果目标是追求视觉上的适当,无需进行$a'=a^{1/\gamma}$的修正。