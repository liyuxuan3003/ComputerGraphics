\section{光栅化三角形}
在这一节,我们要研究的是如何光栅化一个三角形,如\xref{fig:光栅化三角形}所示。应指出的是,光栅化三角形并非意味着对三角形的每条边重复一次光栅化直线,它需要正确的填充三角形内的像素。
\begin{Figure}[光栅化三角形]
    \includegraphics[scale=0.8]{build/Chapter08B_01.fig.pdf}
\end{Figure}
三角形的三个顶点用$\vb{p}_0,\vb{p}_1,\vb{p}_2$表示
\begin{Equation}
    \vb{p}_0=(x_0,y_0)\qquad
    \vb{p}_1=(x_1,y_1)\qquad
    \vb{p}_2=(x_2,y_2)
\end{Equation}
三角形的三条边所在直线可以相应表示为
\begin{Gather}
    f_{01}(x,y)=(y_0-y_1)x+(x_1-x_0)y+x_0y_1-x_1y_0=0\\
    f_{12}(x,y)=(y_1-y_2)x+(x_2-x_1)y+x_1y_2-x_2y_1=0\\
    f_{20}(x,y)=(y_2-y_0)x+(x_0-x_2)y+x_2y_0-x_0y_2=0
\end{Gather}
在填充三角形的过程,我们需要遍历所有可能的像素,并依次判断每个像素是否在三角形的内部。当然,遍历整个屏幕的点显然太多了,我们先可以确定一个能覆盖住整个三角形的矩形区域$[x_{\min},x_{\max}]\times[y_{\min},y_{\max}]$以缩小遍历范围,其中$x_{\min},x_{\max},y_{\min},y_{\max}$由下式给出
\begin{Gather}
    x_{\min}=\te{floor}(\min(x_0,x_1,x_2))\\
    x_{\max}=\te{ceiling}(\max(x_0,x_1,x_2))\\
    y_{\min}=\te{floor}(\min(y_0,y_1,y_2))\\
    y_{\max}=\te{ceiling}(\max(y_0,y_1,y_2))
\end{Gather}

那么,如何确定一个点是否在三角形内呢?我们可以利用该点的重心坐标$\alpha,\beta,\gamma$来判断。我们知道若点在三角形内则$\alpha,\beta,\gamma\in(0,1)$,不过由于$\alpha+\beta+\gamma=1$,从务实的角度我们只需要判断$\alpha,\beta,\gamma>0$即可。现在的问题就是重心坐标$\alpha,\beta,\gamma$如何计算了,考虑以下公式
\begin{Gather}
    \alpha=f_{12}(x,y)/f_{12}(x_0,y_0)\\
    \beta=f_{20}(x,y)/f_{20}(x_1,y_1)\\
    \gamma=f_{01}(x,y)/f_{01}(x_2,y_2)
\end{Gather}
由此可见,重心坐标$\alpha,\beta,\gamma$实际上代表了$(x,y)$相对于$\vb{p}_0,\vb{p}_1,\vb{p}_2$距其对边的距离,以$\alpha$为例,顶点$\vb{p}_0=(x_0,y_0)$的对边是$f_{12}$,现在计算$\alpha=f_{12}(x,y)/f_{12}(x_0,y_0)$,我们注意到
\begin{itemize}
    \item 若$\alpha=0$,则代表$(x,y)$位于$\vb{p}_0$的对边$f_{12}$上。
    \item 若$\alpha=1$,则代表$(x,y)$位于$\vb{p}_0$自身上。
\end{itemize}
然而,由一个潜藏的边界问题,如\xref{fig:光栅化三角形}所示,在$\vb{p}_0$的对边$f_{12}$上,有一个用虚线标注的像素恰好落在了$f_{12}$上,这意味着其$\alpha=0$,按照$\alpha,\beta,\gamma>0$的条件是无法通过的,应当不绘制该像素。然而,假如在三角形$\vb{p}_0,\vb{p}_1,\vb{p}_2$的基础上有一共享边$f_{12}$的邻接三角形$\vb{p}_0',\vb{p}_1,\vb{p}_2$,由于两者都不会绘制落在边上的像素,这就意味着,该像素虽然位于两个三角形中间,却是一个空洞,这显然是不合理的。当然,我们简单的将条件改为$\alpha,\beta,\gamma\geq 0$可以填上这个空洞,但这同样会导致该边线像素被绘制了两次。我们希望找到一种方法,能让边界像素仅被某一侧的三角形绘制一次依次。我们注意到,两个三角形的顶点$\vb{p}_0$和$\vb{p}_0'$分别位于边$f_{12}$的两侧,考虑屏幕外的点$(-1,-1)$作为参照,我们可以通过判断$(-1,-1)$是否与顶点位于顶点对边的同一侧来保证只有一侧的三角形会绘制边线像素。在这个例子中$\vb{p}_0$与$(-1,-1)$位于边$f_{12}$的同一侧,这可以通过$f_{12}(x_0,y_0)f_{12}(-1,-1)>0$判断,因此我们最终决定绘制该边线像素。

请注意,这种处理边线像素能生效的前提是,两侧的三角形对于它们的公共边的方程必须是相同!若一侧是$f_{12}$而另一侧是$f_{21}$,那仍然会两者出现都不绘制或都绘制的情况。\goodbreak

总结来说,完整的判断流程可以用以下三个条件的与表示
\begin{itemize}
    \item $\alpha>0$或$\alpha=0$时有$f_{12}(x_0,y_0)f_{12}(-1,-1)>0$成立。
    \item $\beta>0$或$\beta=0$时有$f_{20}(x_1,y_1)f_{20}(-1,-1)>0$成立。
    \item $\gamma>0$或$\gamma=0$时有$f_{01}(x_2,y_2)f_{01}(-1,-1)>0$成立。
\end{itemize}
接下来的问题是,我们如何确定三角形内的像素应当填充为何种颜色?有两种着色频率
\begin{itemize}
    \item 逐顶点着色(Per-vertex Shading),如\xref{fig:逐顶点着色}所示。
    \item 逐片元着色(Per-fragment Shading),如\xref{fig:逐片元着色}所示。
\end{itemize}
\begin{Figure}[着色频率]
    \begin{FigureSub}[逐顶点着色]
        \includegraphics[scale=0.8]{build/Chapter08D_02.fig.pdf}
    \end{FigureSub}
    \hspace{0.5cm}
    \begin{FigureSub}[逐片元着色]
        \includegraphics[scale=0.8]{build/Chapter08D_03.fig.pdf}
    \end{FigureSub}
\end{Figure}
两者的差别在于,逐顶点着色是先根据顶点法向量$\vb{n}_0,\vb{n}_1,\vb{n}_2$通过\xref{sec:表面着色}介绍的着色方程计算顶点处的颜色$c_0,c_1,c_2$,随后通过插值$c=\alpha c_0+\beta c_1+\gamma c_2$得到三角面内某一特定像素的颜色,这一插值方法被称为重心坐标插值(Barycentric Interpolation)。在\xref{fig:光栅化三角形}中表现的就是逐顶点着色,顶点$\vb{p}_0,\vb{p}_1,\vb{p}_2$被分别假定为红、绿、蓝。然而,我们知道有一些着色方法对于法向量的法向量极为敏感,为此,逐片元着色先对法向量$\vb{n}_0,\vb{n}_1,\vb{n}_2$进行$\vb{n}=\alpha\vb{n}_0+\beta\vb{n}_1+\gamma\vb{n}_2$插值,这样一来三角面内就分布着均匀过渡的法向量,由此再计算颜色,就可以获得更均匀的着色效果。逐片元着色相较逐顶点着色具有更好的效果但也更为耗时,因此我们可以根据具体情况采用合适的着色频率,例如,Blinn-Phong着色对法向量非常敏感,Lambertian着色和Ambient着色对法向量则相对不敏感,两者就可以分别采用逐片元着色和逐顶点着色。

\begin{Figure}[顶点处的法向量]
    \includegraphics[scale=0.8]{build/Chapter08D_01.fig.pdf}
\end{Figure}

但这里尚有一个问题:同一个三角面只有一个法向量,为什么会有三角面顶点的法向量这种提法?如\xref{fig:顶点处的法向量}所示,顶点处的法向量定义为共享该顶点的所有三角面的法向量的平均值。