\section{栅格设备}

\subsection{输出设备}

图像输出设备可以分为两类:显示(输出到屏幕)和打印(输出到纸张)。

显示主要有两种方式
\begin{itemize}
    \item 透射式(Transmissive):LCD显示(Liquid Crystal Display),基于液晶。
    \item 发射式(Emissive):LED显示(Light Emitting Diode),基于发光二极管。
\end{itemize}
显示屏是由固定的像素阵列组成的,每个像素可以控制其发光强度,对于彩色显示,如\xref{fig:彩色显示的原理}所示,其每个像素将具有三个颜色分别为红、绿、蓝的子像素,当距屏幕一定距离时,我们无法分辨出子像素,而是会看到混合出的颜色。这样控制红、绿、蓝的发光强度就可以控制颜色。

\begin{Figure}[彩色显示的原理]
    \includegraphics[scale=0.8]{build/Chapter03A_01.fig.pdf}
\end{Figure}

在LED显示中,每个像素点就是一个发光二极管,这里展示的共阳极接法,即将所有二极管的正端接在一起并加正电压,此时,阴极接负电压则像素点亮,阴极接正电压则像素熄灭。

\begin{Figure}[LED显示]
    \includegraphics[scale=0.8]{build/Chapter03A_04.fig.pdf}
\end{Figure}

在LCD显示中,每个像素点本身不会发光,而是采用背光显示。在整个屏幕的底部有一块背光(Backlight)作为光源,像素将通过阻挡背光通过实现该像素的熄灭。具体来说,光会首先通过水平偏振镜,从非偏振光变为偏振光。随后,水平偏振光将通过液晶。液晶是一种具有特殊性质的物质,当液晶不加电压时,光将直接通过,当液晶加电压时,光在通过时偏振状态会旋转$\pi/2$,这个例子中,光将会从水平偏振变为垂直偏振。在整个屏幕前部,还有一个垂直偏振镜,它会阻挡水平偏振光的通过但不影响垂直偏振光,故只有液晶通电的像素才会被点亮。
\begin{Figure}[LCD显示]
    \includegraphics[scale=0.8]{build/Chapter03A_03.fig.pdf}
\end{Figure}

打印机有很多不同结构的,最常用的是喷墨打印机(Ink Jet Printer),喷墨打印机中有一个载有墨水的墨盒,墨盒的喷嘴可以在电的控制下喷出非常小的墨滴,随着墨盒的移动,最终在纸张上留下图案。彩色打印机需要若干墨水颜色不同的墨盒,墨水在纸张上混合形成各种颜色。

打印机和显示屏的一个区别在于,它没有物理上存在的像素阵列,它的分辨率取决于墨滴可以有多小,因此,和显示屏不同,打印机的分辨率(Resolution)应该使用像素密度而不是像素的总数描述。像素密度(Pixel Denisty)的单位是$\si{dpi}$,即dots per inch。例如称一个打印机的分辨率为$\SI{1200}{dpi}$的意思就是其能在$\SI{1}{inch}\times\SI{1}{inch}$的区域内产生$1200\times 1200$个墨滴。

\subsection{输入设备}
图像输入设备的代表就是相机。相机的核心是图像传感器,类似显示屏,图像传感器也是一个像素阵列,但其每个像素都是一个半导体感光元件。图像传感器主要有两种不同技术
\begin{itemize}
    \item CCD(Charge Coupled Devices)
    \item CMOS(Complimentary Metal Oxide Semiconductor)
\end{itemize}
\begin{Figure}[彩色图像传感器的原理]
    \includegraphics[scale=0.8]{build/Chapter03A_02.fig.pdf}
\end{Figure}
图像传感器实现彩色的方式和显示器类似,在感光元件前加装红、绿、蓝的滤光片,这样一来该感光元件就只能感应该颜色的光了。然而,图像传感器和显示器的一个很大的不同的是,显示器的一个像素包含了三种不同颜色的子像素,图像传感器的一个像素就是一个特定颜色的像素,如\xref{fig:彩色图像传感器的原理}所示。这种滤镜结构称为拜尔滤镜(Bayer Filter),当最终产生图像的时候,由于每个像素只有一个颜色的数据,会参考临近颜色不同的像素的数据补齐颜色信息,该过程称为反马赛克(Demosaicking)。另外,注意到在拜尔滤镜中各颜色像素的数量是不一样的,每个$2\times 2$的像素阵列中,有$1$个红色、$2$个绿色、$1$个蓝色,这是模拟了人眼对绿光最为敏感。
