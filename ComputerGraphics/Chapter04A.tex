\section{光线追踪的基本概念}

光线追踪的基本概念可以用\cref{fig:光线追踪的基本概念}来说明,对于每个像素,我们将会产生一条光线,光线的原点和方向与选取的投影方式有关。光线会在三维空间中穿行,光线第一个遇到的物体就代表该像素“看到”的是这个物体,像素就会被渲染为这个物体的颜色。例如在\cref{fig:光线追踪的基本概念}中光线追踪到的是三角面$T_1$。三角面$T_2$未被光线经过,三角面$T_3$在光线上但$T_1$在其前面先遇到光线了。
\begin{Figure}[光线追踪的基本概念]
    \includegraphics[scale=0.8]{Chapter04B_03.fig.pdf}
\end{Figure}

因此,简单来说,光线追踪可以分为三个步骤
\begin{enumerate}
    \item 光线产生(Ray Generation),确定每个像素对应的光线的原点和方向。
    \item 光线相交(Ray Intersection),确定最近的与光线相交的物体。
    \item 着色(Shading),确定该像素的颜色。
\end{enumerate}