\section{光线相交}
由于实际的三维模型都是由其表面一系列的三角面组成的,因此,我们唯一要处理的和光线相交的物体就是三角面。和光线的表示类似,三角面也可以用参数方程的方式表达
\begin{BoxDefinition}[三角面的参数方程]
    三角面遵循以下参数方程
    \begin{Equation}
        \vb{f}(\beta,\gamma)=\vb{a}+\beta(\vb{b}-\vb{a})+\gamma(\vb{c}-\vb{a})
    \end{Equation}
\end{BoxDefinition}

其中,$\vb{a},\vb{b},\vb{c}$为三角面的三个顶点,$\beta,\gamma$是参数满足$\beta>0,\gamma>0$以及$\beta+\gamma<1$,这个范围是为了保证点位于三角面内部,因为$\vb{a}+\beta(\vb{b}-\vb{a})+\gamma(\vb{c}-\vb{a})$本身表示的是一整个平面。

现在联立$\vb{p}(t)=\vb{f}(\beta,\gamma)$,代入\xref{def:三维直线的参数方程}和\xref{def:三角面的参数方程},

\begin{Equation}&[1]
    \vb{e}+\vb{d}t=\vb{a}+\beta(\vb{b}-\vb{a})+\gamma(\vb{c}-\vb{a})
\end{Equation}
将$\vb{e},\vb{d}$和$\vb{a},\vb{b},\vb{c}$展开
\begin{Align}
    x_e+tx_d&=x_a+\beta(x_b-x_a)+\gamma(x_c-x_a)\\
    y_e+ty_d&=y_a+\beta(y_b-y_a)+\gamma(y_c-y_a)\\
    z_e+tz_d&=z_a+\beta(z_b-z_a)+\gamma(z_c-z_a)
\end{Align}
我们可以写成标准的矩阵形式
\begin{Equation}
    \begin{pmatrix}
        x_a-x_b&x_a-x_c&x_d\\
        y_a-y_b&y_a-y_c&y_d\\
        z_a-z_b&z_a-z_c&z_d\\
    \end{pmatrix}
    \begin{pmatrix}
        \beta\\
        \gamma\\
        t\\
    \end{pmatrix}=
    \begin{pmatrix}
        x_a-x_e\\
        y_a-y_e\\
        z_a-z_e\\
    \end{pmatrix}
\end{Equation}
这等价于
\begin{Equation}
    \begin{pmatrix}
        \vb{a}-\vb{b}&\vb{a}-\vb{c}&\vb{d}
    \end{pmatrix}
    \begin{pmatrix}
        \beta\\
        \gamma\\
        t
    \end{pmatrix}=
    \begin{pmatrix}
        \vb{a}-\vb{e}
    \end{pmatrix}
\end{Equation}
这个方程可以很容易的通过行列式的相关知识求出结果,至此求交点的问题已经被解决。
% \begin{BoxFormula}[光线和三角面的相交]
%     光线和三角面相交,联立方程为
%     \begin{Equation}
%         \begin{pmatrix}
%             \vb{a}-\vb{b}&\vb{a}-\vb{c}&\vb{d}
%         \end{pmatrix}
%         \begin{pmatrix}
%             \beta\\
%             \gamma\\
%             t
%         \end{pmatrix}=
%         \begin{pmatrix}
%             \vb{a}-\vb{e}
%         \end{pmatrix}
%     \end{Equation}
% \end{BoxFormula}
解出方程后,我们将得到一组$t,\beta,\gamma$,如何判读这个结果?首先,若$\beta,\gamma>0$和$\beta+\gamma<1$不成立,则说明该光线与三角面所在平面的交点并不在三角面内部。其次,应验证$t\in[t_0,t_1]$是否成立,通常而言$t_0=0$而$t_1=+\infty$,我们不希望在视野背后的三角面也被渲染。最后,如果光线与多个三角面存在交点,我们应当渲染的是$t$最小的三角面,因为它在视野中最靠前。

% 这里有一个问题,透视投影中$0\leq t\leq 1$是观察点和像平面间的区域,该区域是否要渲染?