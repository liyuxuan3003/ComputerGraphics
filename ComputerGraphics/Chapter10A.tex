\section{卷积的意义}

卷积的概念我们已经很熟悉了,故这里,我们着眼于实际的信号处理中的离散卷积。
\begin{BoxDefinition}[一维卷积]
    一维卷积定义为
    \begin{Equation}
        (a*b)[i]=\Sum[i'=i-r][i+r]a[i']b[i-i']
    \end{Equation}
\end{BoxDefinition}
\begin{BoxDefinition}[二维卷积]
    二维卷积定义为
    \begin{Equation}
        (a*b)[i,j]=\Sum[i'=i-r][i+r]\Sum[j'=j-r][j+r]a[i',j']b[i-i',j-j']
    \end{Equation}
\end{BoxDefinition}
两者对应的实际例子是
\begin{itemize}
    \item 一维卷积可以认为是对音频(声强随采样时间的变化)的处理。
    \item 二维卷积可以认为是对图像(光强随横纵像素的变化)的处理。
\end{itemize}
这一节我们先以一维卷积为例讨论卷积的意义。通常$a[i]$是待处理的音频信号,是一个相当长的数列,而$b[i]$称为卷积核(Convolution Kernal)或卷积滤波器(Convolution Fliter)。它是以$0$为中心具有$2r+1$长度的数列,其中$r$称为卷积核的半径。例如$r=2$的卷积核就应当有$b[-2],b[-1],b[0],b[1],b[2]$五项。如\xref{fig:卷积的直观理解}所示,卷积的实质是,令卷积核从输入信号上移动过去,将卷积核覆盖范围的信号做加权平均后作为当前卷积核中心位置的输出信号,但要注意的是,卷积核的第一位代表的是其覆盖的最后一位信号的权重。因而,从结果上看,输出信号的每一位是该位附近的输入信号的加权平均!而卷积核则会决定加权平均的权重分布。

\begin{Figure}[卷积的直观理解]
    \includegraphics[scale=0.8]{build/Chapter10A_01.fig.pdf}
\end{Figure}

注意到,为了产生和输入信号等长的输出信号,需要令输入信号两侧“扩展”一些
\begin{itemize}
    \item 卷积时,输入信号是否需要进行扩展,应当扩展多少?
    \item 卷积时,输入信号两侧扩展的位应当以何种方式进行填充?
\end{itemize}
对于第一个问题,有三种可能
\begin{itemize}
    \item 完全:卷积核移动至只有一位与信号交叠,输入信号两侧各扩展$2r$,输出信号变长$2r$。
    \item 等长:卷积核中心位于信号内,输入信号两侧各扩展$r$,输出信号等长。
    \item 合法:卷积核全部位于信号内,输入信号不扩展,输出信号减少$2r$。
\end{itemize}
对于第二个问题,有下面几种方案
\begin{itemize}
    \item 裁剪:扩展的位全部填充$0$。
    \item 复制:扩展的位全部填充信号最边缘的位。
    \item 反射:扩展的位沿边缘反射信号边缘附近的位。
    \item 环绕:扩展的位按该信号为周期信号填充。
\end{itemize}
在通常情况下,使用“等长”和“复制”的策略组合可以得到比较符合预期的结果。