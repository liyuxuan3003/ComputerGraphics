\section{隐式函数的混合}

当然,除了最简单的加法混合,还有更复杂的混合方式,本节我们将讨论更多混合。

\subsection{最大值与最小值混合}
\begin{BoxFormula}[最大值混合]
    最大值混合是指
    \begin{Equation}
        f(\vb{p})=\max(f_A(\vb{p})+f_B(\vb{p}))
    \end{Equation}
\end{BoxFormula}

\begin{BoxFormula}[最小值混合]
    最小值混合是指
    \begin{Equation}
        f(\vb{p})=\min(f_A(\vb{p})+f_B(\vb{p}))
    \end{Equation}
\end{BoxFormula}

在\xref{fig:最大值混合}和\xref{fig:最小值混合}中,我们直观看出两者的几何意义是
\begin{itemize}
    \item 最大值混合下$f(\vb{p})=\te{iso}$代表取$f_A(\vb{p})=\te{iso},f_B(\vb{p})=\te{iso}$的并集(并集混合)。
    \item 最小值混合下$f(\vb{p})=\te{iso}$代表取$f_A(\vb{p})=\te{iso},f_B(\vb{p})=\te{iso}$的交集(交集混合)。
\end{itemize}

在本节的示意图中,坐标$(\pm 0.4,0,0)$,取$r=1,\te{iso}=0.5$,采用Soft Objects,由于其恰好满足$g_d(0.5)=0.5$,这意味着$f_A(\vb{p}),f_B(\vb{p})=\te{iso}$均为半径$0.5$的球,在$(\pm 0.1,0,0)$内重叠。

\begin{Figure}[最大值混合]
    \includegraphics[scale=0.8]{build/Chapter16A_03e.fig.pdf}
\end{Figure}

\begin{Figure}[最小值混合]
    \includegraphics[scale=0.8]{build/Chapter16A_03d.fig.pdf}
\end{Figure}

最大值混合和最小值混合的特点是,其构成的曲面具有不光滑的过渡,我们可以很轻易的看到两个部分的分界线。不过有些情况下这是有用的。例如,构造实体几何(Constructive Solid Geometry, CSG)的核心思想是,应用简单几何体,通过布尔运算,形成复杂几何体。隐式建模不仅可以通过$f_i(\vb{p})=\te{iso}$构成曲面,也可以通过$f_i(\vb{p})\geq\te{iso}$构成几何实体,适用于CSG。

\subsection{Ricchi混合}
Ricchi混合也称为超椭圆混合(Super Elliptic Blend),它有一些好的性质。
\begin{BoxFormula}[Ricci混合]
    Ricci混合的定义是
    \begin{Equation}
        f(\vb{p})=\qty(f_A^n(\vb{p})+f_B^n(\vb{p}))^{1/n}
    \end{Equation}
\end{BoxFormula}


当$n=1$时,Ricchi混合会退化为加法混合
\begin{Equation}
    f_A(\vb{p})+f_B(\vb{p})=\qty(f_A^n(\vb{p})+f_B^n(\vb{p}))^{1/n}|_{n=1}
\end{Equation}
当$n\to\pm\infty$时,Ricchi混合会分别趋于最大值混合和最小值混合
\begin{Gather}
    \max(f_A(\vb{p}),f_B(\vb{p}))=\lim_{n\to+\infty}\qty(f_A^n(\vb{p})+f_B^n(\vb{p}))^{1/n}\\
    \min(f_A(\vb{p}),f_B(\vb{p}))=\lim_{n\to-\infty}\qty(f_A^n(\vb{p})+f_B^n(\vb{p}))^{1/n}
\end{Gather}

因此,如\xref{fig:Ricci混合},当$n\in[1,\infty)$,Ricchi混合提供了从相加混合到并集混合的连续过渡。
\begin{Figure}[Ricci混合]
    \begin{FigureSub}[$n=1$;Ricci n=1]
        \includegraphics[scale=0.8]{build/Chapter16A_03a.fig.pdf}
    \end{FigureSub}\\ \vspace{0.25cm}
    \begin{FigureSub}[$n=2$;Ricci n=2]
        \includegraphics[scale=0.8]{build/Chapter16A_03b.fig.pdf}
    \end{FigureSub}\hspace{1cm}
    \begin{FigureSub}[$n=4$;Ricci n=4]
        \includegraphics[scale=0.8]{build/Chapter16A_03c.fig.pdf}
    \end{FigureSub}
\end{Figure}

\subsection{Pasko混合}
Pasko混合中$\alpha\in[-1,1]$给出从交集至并集的过渡(存疑),而$m$决定了曲面是$C^m$连续的。
\begin{BoxFormula}[Pasko混合]
    Pasko混合的定义是
    \begin{Equation}
        \qquad\qquad
        f(\vb{p})=\qty(f_A(\vb{p})+f_B(\vb{p})+\alpha\sqrt{f_A^2(\vb{p})+f_B^2(\vb{p})})\qty(f_A^2(\vb{p})+f_B^2(\vb{p}))^{m/2}
        \qquad\qquad
    \end{Equation}
\end{BoxFormula}

