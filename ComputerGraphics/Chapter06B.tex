\section{矩阵变换的分解}
本节将研究这样一个问题,对于一般的变换矩阵$\vb{A}$,如何将其拆分为若干基本变换的叠加?

在开始前,有必要说明一下几种特殊矩阵的定义。
\begin{BoxDefinition}[对角矩阵]
    对角矩阵(Diagonal Matrix)是指仅具有对角元素的矩阵
    \begin{Equation}
        \vb{A}=\mr{diag}(a_1,a_2)=\begin{pmatrix}
            a_1&0\\
            0&a_2\\
        \end{pmatrix}
    \end{Equation}
\end{BoxDefinition}

\begin{BoxDefinition}[对称矩阵]
    对称矩阵(Symmetric Matrix)是指转置与自身相等的矩阵
    \begin{Equation}
        \vb{A}=\vb{A}^T
    \end{Equation}
\end{BoxDefinition}
显然,对角矩阵一定是一个对称矩阵。

\begin{BoxDefinition}[正交矩阵]
    正交矩阵(Orthogonal Matrix)是指转置与自身的积为单位阵的矩阵
    \begin{Equation}
        \vb{A}^T\vb{A}=\vb{I}
    \end{Equation}
\end{BoxDefinition}
这里有一个很重要的性质:若一个矩阵$\vb{A}$的列是由一组相互正交的单位向量组成,那么该矩阵一定是一个正交矩阵!试想,对于$\vb{A}^T\vb{A}$的乘积结果,其对角元素是一个单位向量与其自身的点积,因而是$1$,其非对角元素是两个正交向量的点积,因而是$0$,这样就有$\vb{A}^T\vb{A}=\vb{I}$。

\subsection{特征值分解}
在继续介绍特征值分解前,我们有必要先回顾一下特征值和特征向量的概念。

对于一个矩阵$\vb{A}$,若能找到向量$\vb{a}$和数$\lambda$使
\begin{Equation}
    \vb{A}\vb{a}=\lambda\vb{a}
\end{Equation}
那么$\vb{a},\lambda$就称为$\vb{A}$的一组特征向量(Eigen Vector)和特征值(Eigen Value)。换言之,若矩阵作用在一个向量上却只会令这个向量伸长或压缩,那么这个向量就是矩阵的特征向量。

很重要的问题是如何寻找特征向量和特征值?我们可以先加入一个$\vb{I}$\setpeq{求解特征向量和特征值}
\begin{Equation}&[1]
    \vb{A}\vb{a}=\lambda\vb{I}\vb{a}
\end{Equation}
移项至左侧
\begin{Equation}&[2]
    \vb{A}\vb{a}-\lambda\vb{I}\vb{a}=0
\end{Equation}
提出矩阵
\begin{Equation}&[3]
    (\vb{A}-\lambda\vb{I})\vb{a}=0
\end{Equation}
该方程成立唯一的条件是$\vb{A}-\lambda\vb{I}$是一个奇异阵,换言之,行列式为$0$。假设$\vb{A}$是$2\times 2$的
\begin{Equation}&[4]
    \begin{vmatrix}
        a_{11}-\lambda&a_{12}\\
        a_{21}&a_{22}-\lambda\\
    \end{vmatrix}=0
\end{Equation}
即
\begin{Equation}&[5]
    \lambda^2-(a_{11}+a_{22})\lambda+(a_{11}a_{22}-a_{12}a_{21})=0
\end{Equation}
这是一个二次方程,故可以解出两个特征值$\lambda$。事实上,对于一个$n\times n$的矩阵,其具有$n$个特征值。由于四次以上的方程无系解析解,故对于$n>4$的矩阵,其特征值只有数值解。


而对于某个已经求出的特征值$\lambda$,也很容易确定对应的特征向量$\vb{a}=(x,y)$,考虑\xrefpeq{4}
\begin{Equation}
    \begin{pmatrix}
        a_{11}-\lambda&a_{12}\\
        a_{21}&a_{22}-\lambda\\
    \end{pmatrix}
    \begin{pmatrix}
        x\\
        y\\
    \end{pmatrix}=
    \begin{pmatrix}
        0\\
        0\\
    \end{pmatrix}
\end{Equation}
这是一个齐次方程,有无穷多组非零$(x,y)$是方程的解,注意到
\begin{Equation}
    (a_{11}-\lambda)x+a_{12}y=a_{21}x+(a_{22}-\lambda)y
\end{Equation}
整理得到
\begin{Equation}
    (a_{11}-a_{21}-\lambda)x=(a_{22}-a_{12}-\lambda)y
\end{Equation}
因此,只要$x,y$符合下面的比例,那它就是$\lambda$对应的特征向量
\begin{Equation}
    \frac{y}{x}=\frac{a_{11}-a_{21}-\lambda}{a_{22}-a_{12}-\lambda}
\end{Equation}
这一点从特征值的定义$\vb{A}\vb{a}=\lambda \vb{a}$上看也很合理:若$\vb{a}$是特征向量,那么$\vb{a}$的任意倍也都是特征向量,毕竟这只不过是在方程两端同乘了一个相同的系数罢了。但通常,我们会选择那个长度为$1$的那个单位特征向量。除此之外,可以证明一个矩阵的特征向量是两两相互正交的。\goodbreak

任何一个矩阵都具有特征向量和特征值,然而,若矩阵是满足对称矩阵,则矩阵自身可以直接用其特征向量和特征值表示出来,这就是所谓的特征值分解!它的具体形式如下
\begin{BoxFormula}[特征值分解]
    特征值分解(Eigen Value Decomposition, EVD)是指,对于对称矩阵$\vb{A}$,有
    \begin{Equation}
        \vb{A}=\vb{R}\vb{S}\vb{R}^T
    \end{Equation}
    矩阵$\vb{R}$是一个正交矩阵,由特征向量$\vb{r}_1,\vb{r}_2$构成
    \begin{Equation}
        \vb{R}=\qty(\vb{r}_1,\vb{r}_2)
    \end{Equation}
    矩阵$\vb{S}$是一个对角矩阵,由特征值$\lambda_1,\lambda_2$构成
    \begin{Equation}
        \vb{S}=\mr{diag}(\lambda_1,\lambda_2)
    \end{Equation}
\end{BoxFormula}

现在我们来解读结果,不过再次之前,我们先要将特殊矩阵和矩阵变换联系起来
\begin{itemize}
    \item 对角矩阵$\vb{S}=\mr{diag}(\lambda_1,\lambda_2)$对应缩放变换,其中$\lambda_1,\lambda_2$代表沿$\vb{x},\vb{y}$的缩放比。
    \item 正交矩阵$\vb{R}=(\vb{r}_1,\vb{r}_2)$对应旋转变换,其代表$\vb{x},\vb{y}$旋转至$\vb{r}_1,\vb{r}_2$方向,$\vb{R}^T$则相反。
\end{itemize}

特征值分解可以视为这样三步变换
\begin{enumerate}
    \item 旋转,矩阵$\vb{R}^T$作用,代表图形由$\vb{r}_1,\vb{r}_2$方向分别旋转至$\vb{x},\vb{y}$方向。
    \item 缩放,矩阵\hspace{0.45em}$\vb{S}$\hspace{0.45em}作用,代表图形沿$\vb{x},\vb{y}$方向分别缩放$\lambda_1,\lambda_2$倍。
    \item 旋转,矩阵$\vb{R}$作用,代表图形由$\vb{x},\vb{y}$方向分别旋转至$\vb{r}_1,\vb{r}_2$方向。
\end{enumerate}
特征值分解告诉我们,任何对称矩阵代表的变换,本质上都可以视为由“旋转--缩放--旋转”这样三步基本变换的组合变换。更进一步的说,由于缩放沿$\vb{x},\vb{y}$进行,且缩放前后的旋转分别是从特征向量方向旋转至$\vb{x},\vb{y}$再旋转回去,因此,对称矩阵的变换的效果其实就是在特征向量$\vb{r}_1,\vb{r}_2$的方向上分别进行特征值$\lambda_1,\lambda_2$倍的缩放!这是一个非常深刻且直观的几何理解。

\subsection{奇异值分解}
奇异值和奇异值分解的想法是从特征值分解延拓出来的。特征值分解存在一个明显的缺陷,即只有对角矩阵才能适用。奇异值分解期望将任何矩阵都拆分为“旋转--缩放--旋转”的形式,不过势必要做出一些让步,需要将$\vb{A}=\vb{R}\vb{S}\vb{R}^T$需改写为$\vb{A}=\vb{U}\vb{S}\vb{V}^T$,即首尾两个正交矩阵不再是同一个。将左侧$\vb{U}=(\vb{u}_1,\vb{u}_2)$和右侧$\vb{v}=(\vb{v}_1,\vb{v}_2)$中的向量分别称为左奇异向量和右奇异向量(Singular Vector),同时,将$\vb{S}=\mr{diag}(\sigma_1,\sigma_2)$中的数称为奇异值(Singular Value)。

\begin{BoxFormula}[奇异值分解]
    奇异值分解(Singular Value Decomposition, SVD)是指,对于矩阵$\vb{A}$,有
    \begin{Equation}
        \vb{A}=\vb{U}\vb{S}\vb{V}^T
    \end{Equation}
    矩阵$\vb{U},\vb{V}$是正交矩阵,由左奇异向量$\vb{u}_1,\vb{u}_2$和右奇异向量$\vb{v}_1,\vb{v}_2$构成
    \begin{Equation}
        \vb{U}=\qty(\vb{u}_1,\vb{u}_2)\qquad
        \vb{V}=\qty(\vb{v}_1,\vb{v}_2)
    \end{Equation}
    矩阵$\vb{S}$是一个对角矩阵,由特征值$\sigma_1,\sigma_2$构成
    \begin{Equation}
        \vb{S}=\mr{diag}(\sigma_1,\sigma_2)
    \end{Equation}
\end{BoxFormula}

奇异值分解可以视为这样三步变换
\begin{itemize}
    \item 旋转,矩阵$\vb{V}^T$作用,代表图形由$\vb{v}_1,\vb{v}_2$方向分别旋转至$\vb{x},\vb{y}$方向。
    \item 缩放,矩阵\hspace{0.45em}$\vb{S}$\hspace{0.45em}作用,代表图形沿$\vb{x},\vb{y}$方向分别缩放$\sigma_1,\sigma_2$倍。
    \item 旋转,矩阵$\vb{U}$作用,代表图形由$\vb{x},\vb{y}$方向分别旋转至$\vb{u}_1,\vb{u}_2$方向。
\end{itemize}

奇异值分解和特征值分解能如何联系在一起呢?考虑以下矩阵
\begin{Equation}
    \vb{M}=\vb{A}\vb{A}^T
\end{Equation}
代入$\vb{A}=\vb{U}\vb{S}\vb{V}^T$
\begin{Equation}
    \vb{M}=(\vb{U}\vb{S}\vb{V}^T)(\vb{U}\vb{S}\vb{V}^T)^T
\end{Equation}
若干矩阵乘积的转置,等于每个矩阵转置的逆序乘积
\begin{Equation}
    \vb{M}=(\vb{U}\vb{S}\vb{V}^T)(\vb{V}\vb{S}^T\vb{U}^T)
\end{Equation}
由于$\vb{S}$是对角矩阵,故$\vb{S}^T=\vb{S}$
\begin{Equation}
    \vb{M}=(\vb{U}\vb{S}\vb{V}^T)(\vb{V}\vb{S}\vb{U}^T)
\end{Equation}
应用矩阵乘法的分配律
\begin{Equation}
    \vb{M}=\vb{U}\vb{S}(\vb{V}^T\vb{V})\vb{S}\vb{U}^T
\end{Equation}
由于$\vb{V}$是对角矩阵,故$\vb{V}^T\vb{V}=\vb{I}$
\begin{Equation}
    \vb{M}=\vb{U}\vb{S}^2\vb{U}^T
\end{Equation}
这意味着$\vb{M}=\vb{A}^T\vb{A}$必然可以做特征值分解!且此时,特征向量是$\vb{A}$的左奇异向量,特征值是$\vb{A}$的奇异值的平方,因为$\vb{S}^2=\mr{diag}(\sigma_1^2,\sigma_2^2)$。由此可见,通过这种方式,奇异值的计算可以转换为特征值的计算。并且,$\vb{A}^T\vb{A}=\vb{U}\vb{S}^2\vb{U}^T$给出$\vb{U}$,类似的,$\vb{A}\vb{A}^T=\vb{V}\vb{S}^2\vb{V}^T$给出$\vb{V}$。