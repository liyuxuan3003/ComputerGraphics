\section{纹理映射的应用}
更一般的说,纹理映射处理所有在物体上不同位置变化但不真正改变物体形状的性质。

纹理映射的贴图中的数值可以有更丰富的含义,而不仅仅局限于漫反射颜色$c_d$。试想一本书的封面,封面上的书名用烫金字写的,此时,就可以通过纹理映射,在书名处调整镜面反射颜色$c_s$和镜面反射系数$p$使其呈现出金光闪闪的效果。更进一步,纹理映射还可以让平面显示出符合光照的凹凸不平的效果。这有Normal Map和Displacement Map两种方案,前者是通过干预法向量方向伪造表面的起伏,后者则是真正的改变了表面的位置,添加了一个偏移。

纹理映射还可以用于Shadow Map和Environment Map。我们知道,由于光栅化的光照计算在每个表面是独立进行的,因此较难实现涉及光线在多个表面相互作用的效果,例如阴影和反光。但纹理映射可以一定程度上模拟这一点。Shadow Map描述光源在各方向上最近物体的距离,从而在光照计算时确定该物体和光源之间是否有物体遮挡。Environment Map描述物体在各方向观察到的画面,将这些画面作为物体本身的贴图,从而实现物体对周围环境的反光。

% 纹理映射还可以用于Shadow Map和Environment Map。前者即\xref{sec:阴影映射}介绍的阴影映射,此时纹理指的是光源不同方向最近的表面距离,而且不同的是,这种纹理是通过场景生成而非预先设定的。后者可以认为是环境的贴图,即“天空盒”。试想,一款太空游戏需要一个星空背景,由于这种背景仅取决于观察方向,而与观察位置无关,我们可以使用立方体投影或球坐标投影,将一张平面的星空图像映射到背景上,而不用真正把每一颗遥远的星星渲染出来。