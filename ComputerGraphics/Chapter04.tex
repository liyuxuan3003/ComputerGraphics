\chapter{光线追踪}

计算机图形学的一个基本任务就是三维场景的渲染(Redering)。渲染是指,对于一系列排列在三维空间中的物体,产生一张从三维空间特定视角观察这些物体的二维图像。本质上说,渲染是一系列物体为输入以一个像素阵列为输出的过程,无论具体如何实现,渲染总会涉及这样一个问题:每个物体是如何影响每个像素的?渲染有以下两种的主流实现方式
\begin{enumerate}
    \item 物体顺序渲染(Object Order Redering),遍历每个物体,考虑它们会影响哪些像素。
    \item 图像顺序渲染(Image Order Redering),遍历每个像素,考虑它们将如何被物体确定。
\end{enumerate}
若借用计算机语言,渲染即是一个包含遍历物体和遍历像素的双重循环,物体顺序渲染和图像顺序渲染的差别就是哪一循环在外面。两种渲染方式都可以产生相同的图像,但是效率和复杂性是不同的。通常而言,图像顺序渲染更简单也更灵活,不过相较物体顺序渲染需要更长的时间来产生一张图像。本章将研究的光线追踪(Ray Tracking)是一种图像顺序渲染算法。

\section{光线追踪的基本概念}

光线追踪的基本概念可以用\xref{fig:光线追踪的基本概念}来说明,对于每个像素,我们将会产生一条光线,光线的原点和方向与选取的投影方式有关。光线会在三维空间中穿行,光线第一个遇到的物体就代表该像素“看到”的是这个物体,像素就会被渲染为这个物体的颜色。例如在\xref{fig:光线追踪的基本概念}中光线追踪到的是三角面$T_1$。三角面$T_2$未被光线经过,三角面$T_3$在光线上但$T_1$在其前面先遇到光线了。
\begin{Figure}[光线追踪的基本概念]
    \includegraphics[scale=0.8]{build/Chapter04B_03.fig.pdf}
\end{Figure}

因此,简单来说,光线追踪可以分为两个步骤
\begin{enumerate}
    \item 光线产生(Ray Generation),确定每个像素对应的光线的原点和方向。
    \item 光线相交(Ray Intersection),找到最近的与光线相交的物体。
\end{enumerate}
\section{光线产生}
光线产生和投影方式有关,主要有以下两种投影
\begin{itemize}
    \item 平行投影(Parallel Projection),如\cref{fig:平行投影}所示,是指光线从每个像素的中心处发出且所有光线是平行的。平行投影可以分为两种情况,取决于光线方向是否与像平面垂直,分别称为正交投影(Orthographic)和斜交投影(Oblique),其中正交投影比较常见,我们在\cref{fig:平行投影}展示的也是正交投影的情况。平行投影常用于工程图(例如机械和建筑)的绘制,它的最大特征是,三维空间中的平行线经过投影后在二维图像中仍然是平行的。
    \item 透视投影(Perspective Projection),如\cref{fig:透视投影}所示,是指光线从一个特定的观察点处向各个像素的方向发出,观察点位于像平面中心后方处。透视投影符合人眼或相机的成像模式,它的特点是具有透视效应,近大远小,平行线不一定再平行。试想,站在一条笔直的公路中央望向远方,会发现公路两侧边沿会逐渐靠拢,但它们实际是平行的。
\end{itemize}

\begin{Figure}[两种投影方式]
    \figuresub[平行投影]{\includegraphics[scale=0.8]{Chapter04B_01.fig.pdf}}    
    \figuresub[透视投影]{\includegraphics[scale=0.8]{Chapter04B_02.fig.pdf}}    
\end{Figure}

有趣的是,平行投影在一定意义上可以认为是观察点位于无穷远处的透视投影。这一理解最直观的感受是,长焦镜头具有压缩空间的效果,即透视带来的近大远小在长焦下不太明显了。

在\cref{fig:两种投影方式}中,有一些标注需要说明。首先,无论是平行投影还是透视投影,我们都会确定一个原点$\vb{e}$,对于平行投影是像平面中心,对于透视投影是观察点。其次,在$\vb{e}$上标注了三个重要的单位矢量$\vb{u},\vb{v},\vb{w}$,三者呈右手螺旋关系。其中,$\vb{w}$是视野方向的反方向,$\vb{v},\vb{u}$均位于像平面,它们会定义画面的“上”和“右”在空间中的对应方向。最后,$u_l,u_r,v_b,v_t$分别确定了画面在左、右、下、上的边界。特别的,在透视投影中,$d$还用于表示观察点和像平面间的距离。

假设二维图像的像素总数是$n_x\times n_y$,对于$(i,j)$处的像素,其水平位置$u$和垂直位置$v$为
\begin{Gather}
    u=x_l+(x_r-x_l)(i+0.5)/n_x\\
    v=y_b+(y_t-y_b)(j+0.5)/n_y
\end{Gather}

我们可以用一个参数方程表达三维空间的直线(即这里的光线)
\begin{BoxDefinition}[三维直线的参数方程]
    三维空间中的直线遵循以下参数方程
    \begin{Equation}
        \vb{p}(t)=\vb{e}+\vb{d}t
    \end{Equation}
\end{BoxDefinition}

其中,$\vb{e}$代表光线的原点,$\vb{d}$代表光线的方向。应指出的是光线的原点$\vb{e}$未必是\cref{fig:两种投影方式}中投影的原点$\vb{e}$,为强调这一区别,在本小节将光线方程$\vb{p}(t)=\vb{e}+\vb{d}t$中的$\vb{e},\vb{d}$记为$\vb{e}_{ray},\vb{d}_{ray}$。

对于平行投影,光线的原点在变化,但方向不变
\begin{BoxFormula}[平行投影的光线方程]
    平行投影的光线方程,原点$\vb{e}_{ray}$和方向$\vb{d}_{ray}$分别为
    \begin{Gather}
        \vb{e}_{ray}=\vb{e}+u\vb{u}+v\vb{v}\\
        \vb{d}_{ray}=-\vb{w}
    \end{Gather}
\end{BoxFormula}

对于透视投影,光线的方向在变化,但原点不变
\begin{BoxFormula}[透视投影的光线方程]
    透视投影的光线方程,原点$\vb{e}_{ray}$和方向$\vb{d}_{ray}$分别为
    \begin{Gather}
        \vb{e}_{ray}=\vb{e}\\
        \vb{d}_{ray}=-\vb{d}\vb{w}+u\vb{u}+v\vb{v}
    \end{Gather}
\end{BoxFormula}
\section{光线相交}
由于实际的三维模型都是由其表面一系列的三角面组成的,因此,我们唯一要处理的和光线相交的物体就是三角面。和光线的表示类似,三角面也可以用参数方程的方式表达
\begin{BoxDefinition}[三角面的参数方程]
    三角面遵循以下参数方程
    \begin{Equation}
        \vb{f}(\beta,\gamma)=\vb{a}+\beta(\vb{b}-\vb{a})+\gamma(\vb{c}-\vb{a})
    \end{Equation}
\end{BoxDefinition}

其中,$\vb{a},\vb{b},\vb{c}$为三角面的三个顶点,$\beta,\gamma$是参数满足$\beta>0,\gamma>0$以及$\beta+\gamma<1$,这个范围是为了保证点位于三角面内部,因为$\vb{a}+\beta(\vb{b}-\vb{a})+\gamma(\vb{c}-\vb{a})$本身表示的是一整个平面。

现在联立$\vb{p}(t)=\vb{f}(\beta,\gamma)$,代入\xref{def:三维直线的参数方程}和\xref{def:三角面的参数方程},

\begin{Equation}&[1]
    \vb{e}+\vb{d}t=\vb{a}+\beta(\vb{b}-\vb{a})+\gamma(\vb{c}-\vb{a})
\end{Equation}
将$\vb{e},\vb{d}$和$\vb{a},\vb{b},\vb{c}$展开
\begin{Align}
    x_e+tx_d&=x_a+\beta(x_b-x_a)+\gamma(x_c-x_a)\\
    y_e+ty_d&=y_a+\beta(y_b-y_a)+\gamma(y_c-y_a)\\
    z_e+tz_d&=z_a+\beta(z_b-z_a)+\gamma(z_c-z_a)
\end{Align}
我们可以写成标准的矩阵形式
\begin{Equation}
    \begin{pmatrix}
        x_a-x_b&x_a-x_c&x_d\\
        y_a-y_b&y_a-y_c&y_d\\
        z_a-z_b&z_a-z_c&z_d\\
    \end{pmatrix}
    \begin{pmatrix}
        \beta\\
        \gamma\\
        t\\
    \end{pmatrix}=
    \begin{pmatrix}
        x_a-x_e\\
        y_a-y_e\\
        z_a-z_e\\
    \end{pmatrix}
\end{Equation}
这等价于
\begin{Equation}
    \begin{pmatrix}
        \vb{a}-\vb{b}&\vb{a}-\vb{c}&\vb{d}
    \end{pmatrix}
    \begin{pmatrix}
        \beta\\
        \gamma\\
        t
    \end{pmatrix}=
    \begin{pmatrix}
        \vb{a}-\vb{e}
    \end{pmatrix}
\end{Equation}
这个方程可以很容易的通过行列式的相关知识求出结果,至此求交点的问题已经被解决。
% \begin{BoxFormula}[光线和三角面的相交]
%     光线和三角面相交,联立方程为
%     \begin{Equation}
%         \begin{pmatrix}
%             \vb{a}-\vb{b}&\vb{a}-\vb{c}&\vb{d}
%         \end{pmatrix}
%         \begin{pmatrix}
%             \beta\\
%             \gamma\\
%             t
%         \end{pmatrix}=
%         \begin{pmatrix}
%             \vb{a}-\vb{e}
%         \end{pmatrix}
%     \end{Equation}
% \end{BoxFormula}
解出方程后,我们将得到一组$t,\beta,\gamma$,如何判读这个结果?首先,若$\beta,\gamma>0$和$\beta+\gamma<1$不成立,则说明该光线与三角面所在平面的交点并不在三角面内部。其次,应验证$t\in[t_0,t_1]$是否成立,通常而言$t_0=0$而$t_1=+\infty$,我们不希望在视野背后的三角面也被渲染。最后,如果光线与多个三角面存在交点,我们应当渲染的是$t$最小的三角面,因为它在视野中最靠前。

% 这里有一个问题,透视投影中$0\leq t\leq 1$是观察点和像平面间的区域,该区域是否要渲染?