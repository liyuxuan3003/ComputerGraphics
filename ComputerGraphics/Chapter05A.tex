\section{Lambertian着色}

Lambertian着色是最简单的模型,其认为,照射至物体表面的光取决于表面和光源的夹角
\begin{itemize}
    \item 若物体表面正对着光源,则表面是完全照亮的。
    \item 若物体表面平行于光源,则表明是完全黑暗的。
\end{itemize}

\begin{Figure}[Lambertian着色]
    \includegraphics[scale=0.8]{build/Chapter05A_01.fig.pdf}
\end{Figure}

如\xref{fig:Lambertian着色}所示,$\vb{n}$是表面的法向,$\vb{l}$是光源的方向,它们都被设定为单位向量。
\begin{BoxFormula}[Lambertian着色]
    Lambertian着色认为
    \begin{Equation}
        c=c_lc_r\max(0,\vb{n}\cdot\vb{l})
    \end{Equation}
\end{BoxFormula}
其中,$c_l$是光源颜色,$c_r$是物体散射颜色,该公式应当对RGB各计算一次以得到RGB每个分量的结果。在\xref{fig:Lambertian着色}中用$\theta$表示$\vb{n},\vb{l}$的夹角,由于$\vb{n},\vb{l}$都是单位矢量,因此$\vb{n}\cdot\vb{l}=\cos\theta$,这符合前面的预期,当$\theta=0$时给出$1$
即最亮,当$\theta=\pm\pi/2$时给出$0$即最暗。然而我们必须要处理的一种情况是当光源位于表面背后,此时$\vb{n}\cdot\vb{l}<0$为负数,这会违背RGB的数值在区间$[0,1]$的要求,而且直观上“不可能比黑更黑”,因此需要将$\vb{n}\cdot\vb{l}$和$0$之间取$\max$函数。

Lambertian着色的一个特点是它与视角无关。这一点我们可以这样理解:Lambertian着色假定物体表面向各方向均匀散射了从光源接收的能量,所以从各个视角看起来都是一样的。