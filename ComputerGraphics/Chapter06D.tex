\section{仿射变换与平移}
通过前面几节,我们已经很清楚的认识到,在三维空间中的一个线性变换可以写成
\begin{Equation}
    \begin{pmatrix}
        x'\\
        y'\\
        z'\\
    \end{pmatrix}=
    \begin{pmatrix}
        m_{11}&m_{12}&m_{13}\\
        m_{21}&m_{22}&m_{23}\\
        m_{31}&m_{32}&m_{33}
    \end{pmatrix}
    \begin{pmatrix}
        x\\
        y\\
        z\\
    \end{pmatrix}
\end{Equation}
我们可以展开来写
\begin{Gather}
    x'=m_{11}x+m_{12}y+m_{13}z\\
    y'=m_{21}x+m_{22}y+m_{23}z\\
    z'=m_{31}x+m_{32}y+m_{33}z
\end{Gather}
我们通过不同的系数可以用上式表示缩放、剪切、旋转、反射等各种变换,但是,却无法表示最简单的平移!换言之,所有变换都是以$(0,0,0)$为中心进行的。若要考虑平移,则需有
\begin{Gather}
    x'=m_{11}x+m_{12}y+m_{13}z+x_t\\
    y'=m_{21}x+m_{22}y+m_{23}z+y_t\\
    z'=m_{31}x+m_{32}y+m_{33}z+z_t
\end{Gather}
诚然,我们可以将$\vb{r}'=\vb{M}\vb{r}$改写为$\vb{r}'=\vb{M}\vb{r}+\vb{r}_t$来单独处理平移的影响。然而这样做不够优雅,我们希望将平移也囊括在矩阵乘法内,这该怎么做呢?有一个极为巧妙的方法是引入齐次坐标(Homogeneous Coordinates)的概念,简单来说,它为每个向量多加了一个维度
\begin{Equation}
    \begin{pmatrix}
        x\\
        y\\
        z\\
        1\\
    \end{pmatrix}\qquad
    \begin{pmatrix}
        x\\
        y\\
        z\\
        0\\
    \end{pmatrix}
\end{Equation}
这个新增的第四维度的元素可以被填充为$0$和$1$
\begin{itemize}
    \item 添加$1$,代表该向量是一个“位矢”,(绝对位置、方位)。
    \item 添加$0$,代表该向量是一个“位移”,(相对位置、方向)。
\end{itemize}
这个设计是很合理且自然的,试想,两个“位置”的差就自然是一个“位矢”
\begin{Equation}
    \begin{pmatrix}
        x_1\\
        y_1\\
        z_1\\
        1\\
    \end{pmatrix}-
    \begin{pmatrix}
        x_2\\
        y_2\\
        z_2\\
        1\\
    \end{pmatrix}=
    \begin{pmatrix}
        x_1-x_2\\
        y_1-y_2\\
        z_1-z_2\\
        0\\
    \end{pmatrix}
\end{Equation}
而有了齐次坐标的定义后,平移就可以通过矩阵乘一并处理了
\begin{Equation}
    \begin{pmatrix}
        x'\\
        y'\\
        z'\\
        1\\
    \end{pmatrix}=
    \begin{pmatrix}
        m_{11}&m_{12}&m_{13}&x_t\\
        m_{21}&m_{22}&m_{23}&y_t\\
        m_{31}&m_{32}&m_{33}&z_t\\
        0&0&0&1\\
    \end{pmatrix}
    \begin{pmatrix}
        x\\
        y\\
        z\\
        1\\
    \end{pmatrix}
\end{Equation}

要注意,这里是“\textbf{先做线性变换,后做平移变换}”,若拆分开来写,则先作用的在后
\begin{Equation}
    \begin{pmatrix}
        x'\\
        y'\\
        z'\\
        1\\
    \end{pmatrix}=
    \begin{pmatrix}
        1&0&0&x_t\\
        0&1&0&y_t\\
        0&0&1&z_t\\
        0&0&0&1\\
    \end{pmatrix}
    \begin{pmatrix}
        m_{11}&m_{12}&m_{13}&0\\
        m_{21}&m_{22}&m_{23}&0\\
        m_{31}&m_{32}&m_{33}&0\\
        0&0&0&1\\
    \end{pmatrix}
    \begin{pmatrix}
        x\\
        y\\
        z\\
        1\\
    \end{pmatrix}
\end{Equation}

这种包含了平移变换的线性变换称为仿射变换(Affine Transformation)。这里可以看出齐次坐标最妙的一点,它不仅解决了平移,而且,倘若参与仿射变换的齐次坐标中的第四维度元素是$0$而不是$1$,即代表参与变换的是一个“位移”而不是“位矢”,则$x_t,y_t,z_t$代表的平移不会作用,且变换后该位置仍为$0$。这是合理的,因为相对距离不会在整体平移中发生变化!

\begin{BoxDefinition}[仿射变换]
    仿射变换矩阵为
    \begin{Equation}
        \vb{M}=
        \begin{pmatrix}
            m_{11}&m_{12}&m_{13}&x_t\\
            m_{21}&m_{22}&m_{23}&y_t\\
            m_{31}&m_{32}&m_{33}&z_t\\
            0&0&0&1\\
        \end{pmatrix}
    \end{Equation}
\end{BoxDefinition}

仿射变换和齐次坐标极大的简化了变换的表示,之后我们还会进一步拓展齐次坐标的用途。
